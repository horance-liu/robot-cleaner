\begin{savequote}[45mm]
\ascii{Write programs for people first, computers second.}
\qauthor{\ascii{- Steve McConnell}}
\end{savequote}

\chapter{MoveOn} 
\label{ch:move-on}

\section{需求}

\begin{content}

\begin{enum}
\eitem{当\ascii{Robot}收到\ascii{FORWARD}指令后,\ascii{Robot}保持方向,并向前移动一个坐标;}
\eitem{当\ascii{Robot}收到\ascii{BACKWARD}指令后,\ascii{Robot}保持方向,并向后移动一个坐标;}
\end{enum}

例如:
\begin{enum}
\eitem{\ascii{Robot}原来在\ascii{(0, 0, N)},执行\ascii{FORWARD}指令后,新的位置在\ascii{(0, 1, N)};}
\eitem{\ascii{Robot}原来在\ascii{(0, 0, N)},执行\ascii{BACKWARD}指令后,新的位置在\ascii{(0, -1, N)};}
\end{enum}

\end{content}

\section{forward指令}

\begin{content}

\subsection{测试用例}

\begin{leftbar}
\begin{c++}[caption={test/robot-cleaner/TestRobotCleaner.h}]
TEST("forward instruction")
{
    WHEN_I_send_instruction(FORWARD);
    THEN_the_robot_cleaner_should_be_in(Position(0, 1, NORTH));
}
\end{c++}
\end{leftbar}

\subsection{引入Command模式}

为了试图通过测试用例,简单伪实现\ascii{RobotCleaner::exec}。

\begin{tabular}{@{}p{\dimexpr0.45\linewidth-\tabcolsep} 
                 | p{\dimexpr0.45\linewidth-\tabcolsep}@{}}
\begin{c++}[caption={src/robot-cleaner/RobotCleaner.cpp}]
void RobotCleaner::exec(Instruction instruction)
{
    position.turnTo(instruction == LEFT);
}
\end{c++}
&
\begin{c++}[caption={src/robot-cleaner/RobotCleaner.cpp}]
void RobotCleaner::exec(Instruction instruction)
{
    if (instruction == FORWARD)
    {
        position = Position(0, 1, NORTH);
        return;
    }

    position.turnTo(instruction == LEFT);
}
\end{c++}
\end{tabular}

测试虽然通过了,但设计已经变得无法容忍的境地了。此时需要分离出\ascii{Instruction}的变化方向,将它与执行指令的主体\ascii{RobotCleaner}相分离。为此引入\ascii{Command}模式,将指令执行的算法封装在独立的对象中,从而方便地支持指令集的扩展。

首先将\ascii{Instruction}从枚举升级为接口类;为了快速恢复测试的运行,暂时保留诸如\ascii{LEFT, RIGHT, FOWARD}等常量。

\begin{tabular}{@{}p{\dimexpr0.45\linewidth-\tabcolsep} 
                 | p{\dimexpr0.45\linewidth-\tabcolsep}@{}}
\begin{c++}[caption={include/robot-cleaner/Instruction.h}]
#ifndef HC37C3D94_43F1_4677_BD56_34CB78EFEC75
#define HC37C3D94_43F1_4677_BD56_34CB78EFEC75

struct Instruction { LEFT, RIGHT, FORWARD };

#endif
\end{c++}
&
\begin{c++}[caption={include/robot-cleaner/Instruction.h}]
#ifndef HC37C3D94_43F1_4677_BD56_34CB78EFEC75
#define HC37C3D94_43F1_4677_BD56_34CB78EFEC75

#include "base/Role.h"

struct Position;

DEFINE_ROLE(Instruction)
{
    ABSTRACT(void exec(Position& position));
};

Instruction* left();
Instruction* right();
Instruction* forward();

#define LEFT    left()
#define RIGHT   right()
#define FORWARD forward()

#endif
\end{c++}
\end{tabular}

修改原来对\ascii{Instruction}的所有调用,包括测试用例,此刻需要传递指针类型了;另外,还需要修改\ascii{RobotCleaner::exec}的接口设计和实现。

\begin{tabular}{@{}p{\dimexpr0.45\linewidth-\tabcolsep} 
                 | p{\dimexpr0.45\linewidth-\tabcolsep}@{}}
\begin{c++}[caption={src/robot-cleaner/RobotCleaner.cpp}]
#include "robot-cleaner/RobotCleaner.h"
#include "robot-cleaner/Instruction.h"

void RobotCleaner::exec(Instruction instruction)
{
    if (instruction == FORWARD)
    {
        position = Position(0, 1, NORTH);
        return;
    }

    position.turnTo(instruction == LEFT);
}
\end{c++}
&
\begin{c++}[caption={src/robot-cleaner/RobotCleaner.cpp}]
#include "robot-cleaner/RobotCleaner.h"
#include "robot-cleaner/Instruction.h"

void RobotCleaner::exec(Instruction* instruction)
{
    instruction->exec(position);
    delete instruction;
}
\end{c++}
\end{tabular}

此刻链接是失败的,为了快速链接和测试用例,创建\ascii{Instruction.cpp}的文件,实现所有的工厂方法,并将\ascii{ForwardInstruction}暂时进行伪实现。

\begin{leftbar}
\begin{c++}[caption={src/robot-cleaner/Instruction.cpp}]
#include "robot-cleaner/Instruction.h"
#include "robot-cleaner/Position.h"

namespace
{
    struct TurnToInstruction : Instruction
    {
        explicit TurnToInstruction(bool left)
          : left(left)
        {}

    private:
        OVERRIDE(void exec(Position& position))
        {
            position.turnTo(left);
        }

    private:
        bool left;
    };
}

Instruction* left()
{ return new TurnToInstruction(true); }

Instruction* right()
{ return new TurnToInstruction(false); }

namespace
{
    struct ForwardInstruction : Instruction
    {
        OVERRIDE(void exec(Position& position))
        {
            position = Position(0, 1, NORTH);
        }
    };
}

Instruction* forward()
{ return new ForwardInstruction; }
\end{c++}
\end{leftbar}

\subsection{消除重复}

\ascii{Instruction}中指令存在重复,例如\ascii{left}和\ascii{LEFT}事实上是同一个概念。宏定义的存在主要是为了让测试尽快通过而暂时保留的产物。

\begin{c++}[caption={include/robot-cleaner/Instruction.h}]
#ifndef HC37C3D94_43F1_4677_BD56_34CB78EFEC75
#define HC37C3D94_43F1_4677_BD56_34CB78EFEC75

#include "base/Role.h"

struct Position;

DEFINE_ROLE(Instruction)
{
    ABSTRACT(void exec(Position& position));
};

Instruction* left();
Instruction* right();
Instruction* forward();

#define LEFT    left()
#define RIGHT   right()
#define FORWARD forward()

#endif
\end{c++}

首先直接使用工厂方法替代宏定义表示指令,确保测试快速通过。

\begin{tabular}{@{}p{\dimexpr0.45\linewidth-\tabcolsep} 
                 | p{\dimexpr0.45\linewidth-\tabcolsep}@{}}
\begin{c++}[caption={test/robot-cleaner/TestRobotCleaner.h}]
TEST("left instruction")
{
    WHEN_I_send_instruction(LEFT);
    THEN_the_robot_cleaner_should_be_in(Position(0, 0, WEST));
}

TEST("right instruction")
{
    WHEN_I_send_instruction(RIGHT);
    THEN_the_robot_cleaner_should_be_in(Position(0, 0, EAST));
}

TEST("forward instruction")
{
    WHEN_I_send_instruction(FORWARD);
    THEN_the_robot_cleaner_should_be_in(Position(0, 1, NORTH));
}
\end{c++}
&
\begin{c++}[caption={test/robot-cleaner/TestRobotCleaner.h}]
TEST("left instruction")
{
    WHEN_I_send_instruction(left());
    THEN_the_robot_cleaner_should_be_in(Position(0, 0, WEST));
}

TEST("right instruction")
{
    WHEN_I_send_instruction(right());
    THEN_the_robot_cleaner_should_be_in(Position(0, 0, EAST));
}

TEST("forward instruction")
{
    WHEN_I_send_instruction(forward());
    THEN_the_robot_cleaner_should_be_in(Position(0, 1, NORTH));
}
\end{c++}
\end{tabular}

然后从\ascii{Instruction.h}删除指令的所有宏定义,运行测试保证通过。

\begin{tabular}{@{}p{\dimexpr0.45\linewidth-\tabcolsep} 
                 | p{\dimexpr0.45\linewidth-\tabcolsep}@{}}
\begin{c++}[caption={include/robot-cleaner/Instruction.h}]
#ifndef HC37C3D94_43F1_4677_BD56_34CB78EFEC75
#define HC37C3D94_43F1_4677_BD56_34CB78EFEC75

#include "base/Role.h"

struct Position;

DEFINE_ROLE(Instruction)
{
    ABSTRACT(void exec(Position& position));
};

Instruction* left();
Instruction* right();
Instruction* forward();

#define LEFT    left()
#define RIGHT   right()
#define FORWARD forward()

#endif
\end{c++}
&
\begin{c++}[caption={include/robot-cleaner/Instruction.h}]
#ifndef HC37C3D94_43F1_4677_BD56_34CB78EFEC75
#define HC37C3D94_43F1_4677_BD56_34CB78EFEC75

#include "base/Role.h"

struct Position;

DEFINE_ROLE(Instruction)
{
    ABSTRACT(void exec(Position& position));
};

Instruction* left();
Instruction* right();
Instruction* forward();

#endif
\end{c++}
\end{tabular}

\subsection{实现forward指令}

先消除\ascii{ForwardInstruction::exec}的伪实现, 将\ascii{ForwardInstruction::exec}委托给\ascii{Position}。

\begin{tabular}{@{}p{\dimexpr0.45\linewidth-\tabcolsep} 
                 | p{\dimexpr0.45\linewidth-\tabcolsep}@{}}
\begin{c++}[caption={src/robot-cleaner/Instruction.cpp}]
namespace
{
    struct ForwardInstruction : Instruction
    {
        OVERRIDE(void exec(Position& position))
        {
            position = Position(0, 0, E);
        }
    };
}
\end{c++}
&
\begin{c++}[caption={src/robot-cleaner/Instruction.cpp}]
namespace
{
    struct ForwardInstruction : Instruction
    {
        OVERRIDE(void exec(Position& position))
        {
            position.forward();
        }
    };
}
\end{c++}
\end{tabular}

为了让测试尽快通过,然后对\ascii{Position::forward}进行伪实现。

\begin{leftbar}
\begin{c++}[caption={src/robot-cleaner/Position.cpp}]
void Position::forward()
{
    y += 1;
}
\end{c++}
\end{leftbar}

\subsubsection{测试用例}

\begin{leftbar}
\begin{c++}[caption={test/robot-cleaner/TestRobotCleaner.h}]
TEST("left, forward instructions")
{
    WHEN_I_send_instruction(left());
    WHEN_I_send_instruction(forward());
    
    THEN_the_robot_cleaner_should_be_in(Position(-1, 0, WEST));
}
\end{c++}
\end{leftbar}

然后对\ascii{Position::forward}进行伪实现,简单地使用\ascii{switch-case}实现。

\begin{leftbar}
\begin{c++}[caption={src/robot-cleaner/Position.cpp}]
void Position::forward()
{
    switch (orientation)
    {
    case NORTH:
        x += 0; y += 1;
        break;
    case WEST:
        x -= 1; y += 0;
        break;
    default: break;
    }
}
\end{c++}
\end{leftbar}

测试通过了,重构非常成功。循环这个微小的\ascii{TDD}环,将场景列举一下,将\ascii{Position::forward}用最笨的方法驱动实现出来。

\begin{leftbar}
\begin{c++}[caption={test/robot-cleaner/TestRobotCleaner.h}]
TEST("2-left, forward instructions")
{
    WHEN_I_send_instruction(left());
    AND_I_send_instruction(left());
    AND_I_send_instruction(forward());
    
    THEN_the_robot_cleaner_should_be_in(Position(0, -1, SOUTH));
}

TEST("3-left, forward instructions")
{
    WHEN_I_send_instruction(left());
    AND_I_send_instruction(left());
    AND_I_send_instruction(left());
    AND_I_send_instruction(forward());
    
    THEN_the_robot_cleaner_should_be_in(Position(1, 0, EAST));
}

TEST("4-left, forward instructions")
{
    WHEN_I_send_instruction(left());
    AND_I_send_instruction(left());
    AND_I_send_instruction(left());
    AND_I_send_instruction(left());
    AND_I_send_instruction(forward());
    
    THEN_the_robot_cleaner_should_be_in(Position(0, 1, NORTH));
}
\end{c++}
\end{leftbar}

最终,\ascii{Position::forward}的实现如下,的确很不光彩,但这也让我信心倍增,毕竟我通过了所有的测试。

\begin{leftbar}
\begin{c++}[caption={src/robot-cleaner/Position.cpp}]
void Position::forward()
{
    switch (orientation)
    {
    case EAST:
        x += 1; y += 0;
        break;
    case SOUTH:
        x += 0; y -= 1;
        break;
    case WEST:
        x -= 1; y += 0;
        break;
    case NORTH:
        x += 0; y += 1;
        break;
    default: break;
    }
}
\end{c++}
\end{leftbar}

测试通过后,接下来将这个刺眼的\ascii{Position::forward}重构掉。类似于\ascii{turnTo}的实现,\ascii{forward}也存在一个简单、有效的算法,来替换这个脆弱的\ascii{switch-case}实现。

\begin{tabular}{@{}p{\dimexpr0.45\linewidth-\tabcolsep} 
                 | p{\dimexpr0.45\linewidth-\tabcolsep}@{}}
\begin{c++}[caption={src/robot-cleaner/Position.cpp}]
void Position::forward()
{
    switch (orientation)
    {
    case EAST:
        x += 1; y += 0;
        break;
    case SOUTH:
        x += 0; y -= 1;
        break;
    case WEST:
        x -= 1; y += 0;
        break;
    case NORTH:
        x += 0; y += 1;
        break;
    default: break;
    }
}
\end{c++}
&
\begin{c++}[caption={src/robot-cleaner/Position.cpp}]
namespace
{
    const int offsets[] = { 1, 0, -1, 0 };
}

void Position::forward()
{
    x += offsets[orientation];
    y += offsets[3 - orientation];
}
\end{c++}
\end{tabular}

测试通过了,重构非常成功,终于将\ascii{switch-case}干掉了。

\end{content}

\section{backward指令}

\begin{content}

\subsection{快速实现}

\ascii{backward}与\ascii{forward}逻辑差不多,它与\ascii{forward}具有逻辑反的关系。鉴于\ascii{forward}的经验,每次一个微小的\ascii{TDD}循环,小步快跑地实现了\ascii{backward}指令。

\begin{leftbar}
\begin{c++}[caption={test/robot-cleaner/TestRobotCleaner.h}]
TEST("backward instruction")
{
    WHEN_I_send_instruction(backward());
    THEN_the_robot_cleaner_should_be_in(Position(0, -1, NORTH));
}

TEST("1-left, backward instruction")
{
    WHEN_I_send_instruction(left());
    AND_I_send_instruction(backward());
    
    THEN_the_robot_cleaner_should_be_in(Position(1, 0, WEST));
}

TEST("2-left, backward instruction")
{
    WHEN_I_send_instruction(left());
    AND_I_send_instruction(left());
    AND_I_send_instruction(backward());
  
    THEN_the_robot_cleaner_should_be_in(Position(0, 1, SOUTH));
}

TEST("3-left, backward instruction")
{
    WHEN_I_send_instruction(left());
    AND_I_send_instruction(left());
    AND_I_send_instruction(left());
    AND_I_send_instruction(backward());

    THEN_the_robot_cleaner_should_be_in(Position(-1, 0, EAST));
}

TEST("4-left, backward instruction")
{
    WHEN_I_send_instruction(left());
    AND_I_send_instruction(left());
    AND_I_send_instruction(left());
    AND_I_send_instruction(left());
    AND_I_send_instruction(backward());

    THEN_the_robot_cleaner_should_be_in(Position(0, -1, NORTH));
}
\end{c++}
\end{leftbar}

快速实现\ascii{backward}关键字,并将职责委托给\ascii{Position::backward}。

\begin{leftbar}
\begin{c++}[caption={src/robot-cleaner/Instruction.cpp}]
namespace
{
    struct BackwardInstruction : Instruction
    {
        OVERRIDE(void exec(Position& position))
        {
            position.backward();
        }
    };
}

Instruction* backward()
{
    return new BackwardInstruction;
}
\end{c++}
\end{leftbar}

按照\ascii{forward}的实现,\ascii{backward}也很容易实现,因为它只是\ascii{forward}的逻辑反。

\begin{leftbar}
\begin{c++}[caption={src/robot-cleaner/Position.cpp}]
namespace
{
    const int offsets[] = { 1, 0, -1, 0 };
}

void Position::forward()
{
    x += offsets[orientation];
    y += offsets[3 - orientation];
}

void Position::backward()
{
    x -= offsets[orientation];
    y -= offsets[3 - orientation];
}
\end{c++}
\end{leftbar}

至此测试通过了。

\subsection{消除重复}

接下来需要处理将\ascii{Forward}与\ascii{Backward}之间的重复设计,重构第一步先将二者合并为\ascii{MoveOn}操作。

\begin{tabular}{@{}p{\dimexpr0.45\linewidth-\tabcolsep} 
                 | p{\dimexpr0.45\linewidth-\tabcolsep}@{}}
\begin{c++}[caption={src/robot-cleaner/Instruction.cpp}]
namespace
{
    struct ForwardInstruction : Instruction
    {
        OVERRIDE(void exec(Position& position))
        {
            position.forward();
        }
    };
}

Instruction* forward()
{
    return new ForwardInstruction;
}

namespace
{
    struct BackwardInstruction : Instruction
    {
        OVERRIDE(void exec(Position& position))
        {
            position.backward();
        }
    };
}

Instruction* backward()
{
    return new BackwardInstruction;
}
\end{c++}
&                 
\begin{c++}[caption={src/robot-cleaner/Instruction.cpp}]
namespace
{
    struct MoveOnInstruction : Instruction
    {
        explicit MoveOnInstruction(bool forward)
          : step(forward ? 1 : -1)
        {}
    
    private:
        OVERRIDE(void exec(Position& position))
        {
            position.moveOn(step);
        }
        
    private:
        int step;
    };
}

Instruction* forward()
{ return new MoveOnInstruction(true); }

Instruction* backward()
{ return new MoveOnInstruction(false); }
\end{c++}
\end{tabular}

同样地,将\ascii{Position}中原来的\ascii{forward}与\ascii{backward}接口合并为一个\ascii{moveOn}接口。

\begin{tabular}{@{}p{\dimexpr0.45\linewidth-\tabcolsep} 
                 | p{\dimexpr0.45\linewidth-\tabcolsep}@{}}
\begin{c++}[caption={src/robot-cleaner/Position.cpp}]
void Position::forward()
{
    x += offsets[orientation];
    y += offsets[3 - orientation];
}

void Position::backward()
{
    x -= offsets[orientation];
    y -= offsets[3 - orientation];
}
\end{c++}
&
\begin{c++}[caption={src/robot-cleaner/Position.cpp}]
void Position::moveOn(int step)
{
    x += step*offsets[orientation];
    y += step*offsets[3-orientation];
}
\end{c++}
\end{tabular}

\subsection{改善表达力}

\ascii{offsets[3-orientation]}的逻辑有些晦涩,一眼看不出来这是什么东西。第一个感觉给它加一个注释,但最好的策略是改善代码本身的可读性。

事实上,\ascii{offsets[orientation]}其实表示的是:当面朝\ascii{orientation}时,\ascii{Robot}执行\ascii{forward}指令时,在\ascii{X}轴上的偏移量;\ascii{offsets[3-orientation]}则表示:当面朝\ascii{orientation}时,\ascii{Robot}执行\ascii{forward}指令时,在\ascii{Y}轴上的偏移量。

\begin{tabular}{@{}p{\dimexpr0.45\linewidth-\tabcolsep} 
                 | p{\dimexpr0.45\linewidth-\tabcolsep}@{}}
\begin{c++}[caption={src/robot-cleaner/Position.cpp}]
void Position::moveOn(int step)
{
    x += step*offsets[orientation];
    y += step*offsets[3-orientation];
}
\end{c++}
&
\begin{c++}[caption={src/robot-cleaner/Position.cpp}]
inline int Position::getOffsetOfX()
{ 
    return offsets[orientation]; 
}

inline int Position::getStepOfY()
{ 
    return offsets[3 - orientation]; 
}

void Position::moveOn(int step)
{
    x += step*getStepOfX();
    y += step*getStepOfY();
}
\end{c++}
\end{tabular}

至此测试通过,重构完毕。

\end{content}

\section{遗留问题}

\begin{content}

至此,\ascii{MoveOn}已经实现了,但依然存在一些明显的坏味道,将其记录在\ascii{to-do list}中,后续再做改善。

\begin{enum}
\eitem{\ascii{Position}是一个典型的\ascii{Value Object},\ascii{turnTo}的实现具有副作用;}
\eitem{\ascii{Orientation}的枚举成员顺序,\ascii{Position.cpp}中定义的\ascii{orientations, offsets}数组元素的顺序,它们都是“方向顺序”这一知识的三种不同表现形式,具有设计的重复性;更为严重的是\ascii{turnTo, moveOn}的算法实现强依赖于此约定,设计具有脆弱性;}
\eitem{\ascii{turnTo, moveOn}具有严重的传染性,让设计变得更加耦合;}
\eitem{测试用例存在重复设计。}
\end{enum}

\end{content}

