\begin{savequote}[45mm]
\ascii{I'm not a great programmer; I'm just a good programmer with great habits.}
\qauthor{\ascii{- Kent Beck}}
\end{savequote}

\chapter{TurnTo} 
\label{ch:turn-to}

\section{需求}

\begin{content}

\begin{enum}
\eitem{当\ascii{Robot}收到\ascii{LEFT}指令后,\ascii{Robot}向左转\ascii{90}度;}
\eitem{当\ascii{Robot}收到\ascii{RIGHT}指令后,\ascii{Robot}向右转\ascii{90}度;}
\end{enum}

例如:
\begin{enum}
\eitem{\ascii{Robot}原来在\ascii{(0, 0, N)},执行\ascii{LEFT}指令后,新的位置在\ascii{(0, 0, W)};}
\eitem{\ascii{Robot}原来在\ascii{(0, 0, N)},执行\ascii{RIGHT}指令后,新的位置在\ascii{(0, 0, E)};}
\end{enum}

\end{content}

\section{破冰之旅}

\begin{content}

\subsection{第一个测试用例}

从第一个测试用例开始,\ascii{Robot}初始位置应该在{(0, 0, N)}。

\begin{leftbar}
\begin{c++}[caption={test/robot-cleaner/TestRobotCleaner.h}]
#include "testngpp/testngpp.hpp"

FIXTURE(RobotCleaner)
{
    RobotCleaner robot;

    TEST("at the beginning, the robot should be in at the initial position")
    {
        ASSERT_EQ(Position(0, 0, NORTH), robot.getPosition());
    }
};
\end{c++}
\end{leftbar}

\subsection{通过编译}

为了试图通过编译,先创建一个枚举类型\ascii{Orientation},用于表示方向。

\begin{leftbar}
\begin{c++}[caption={include/robot-cleaner/Orientation.h}]
#ifndef H9811B75A_15B3_4DF0_91B7_483C42F74473
#define H9811B75A_15B3_4DF0_91B7_483C42F74473

enum Orientation { EAST, SOUTH, WEST, NORTH };

#endif
\end{c++}
\end{leftbar}

使用随机的生成的宏保护符,使得重命名类和文件名变得极为便捷。

另外为\ascii{Orientation}新建一个头文件,即使它仅仅只有一行代码;而没有将所有代码实体都塞在一个头文件内,制造一个巨型头文件。如此实现给设计带来了诸多的好处:
\begin{enum}
\eitem{领域内的的\ascii{Orientation}概念,提供了更好的可复用性;}
\eitem{职责更单一,从而可以得到更好的物理依赖;}
\eitem{避免了上帝头文件,多人协作并行使用\ascii{GIT, SVN}等代码管理工具时,冲突的概率极大地被减低。}
\end{enum}

然后再创建一个\ascii{Position}类,用于描述\ascii{Robot}所在的地理位置信息。

\begin{leftbar}
\begin{c++}[caption={include/robot-cleaner/Position.h}]
#ifndef H722C49C6_D285_4885_88D1_97A11D669EE1
#define H722C49C6_D285_4885_88D1_97A11D669EE1

#include "robot-cleaner/Orientation.h"

struct Position
{
    Position(int x, int y, Orientation orientation);

    bool operator==(const Position&) const;
};

#endif
\end{c++}
\end{leftbar}

是的,现在最紧急的目标是让代码编译通过,没有考虑\ascii{Position}是怎么实现的,从而没有考虑\ascii{Position}应该包含哪些成员变量。

此外,Position重载了操作符\ascii{==},使得它更加具有\ascii{Value Object}的特性;另外\ascii{Position.h}直接包含了\ascii{Orientation.h},让\ascii{Position.h}具有自满足性,从而使用户无需为包含关系而烦恼。

最后创建一个\ascii{RobotCleaner}类,用于描述\ascii{Robot}主体。就目前测试用例而言,默认构造函数即可,附加提供一个\ascii{getPosition}成员函数即可。

\begin{leftbar}
\begin{c++}[caption={include/robot-cleaner/RobotCleaner.h}]
#ifndef H632C2FEA_66F0_4B68_B047_121B6CE1482E
#define H632C2FEA_66F0_4B68_B047_121B6CE1482E

struct Position;

struct RobotCleaner
{
    Position getPosition() const;
};

#endif
\end{c++}
\end{leftbar}

因为\ascii{getPosition}是一个查询函数,所以它被声明为\ascii{const}成员函数;此外,\ascii{Position}仅作为返回值,前置声明即可,无需包含它的头文件,从而降低了物理依赖性。

\subsection{通过链接}

为了尽快看到\ascii{make}的反馈,实现\ascii{RobotCleaner::getPosition},及其\ascii{Position}类时,直接做了快速的伪实现,毕竟现在的目标只是让链接通过而已。

\begin{leftbar}
\begin{c++}[caption={src/robot-cleaner/RobotCleaner.cpp}]
#include "robot-cleaner/RobotCleaner.h"
#include "robot-cleaner/Position.h"

Position RobotCleaner::getPosition() const
{
    return Position(0, 0, NORTH);
}
\end{c++}
\end{leftbar}

\begin{leftbar}
\begin{c++}[caption={src/robot-cleaner/Position.cpp}]
#include "robot-cleaner/Position.h"

Position::Position(int x, int y, Orientation orientation)
{}

bool Position::operator==(const Position& rhs) const
{
    return false;
}
\end{c++}
\end{leftbar}

\subsection{通过测试}

此刻链接已经通过,但测试运行失败。为了尽快让测试通过,也只需将\ascii{Position}的比较逻辑快速实现。

\begin{minipage}[t]{0.45\linewidth}
\begin{c++}[caption={include/robot-cleaner/Position.h}]
#ifndef H722C49C6_D285_4885_88D1_97A11D669EE1
#define H722C49C6_D285_4885_88D1_97A11D669EE1

#include "robot-cleaner/Orientation.h"

struct Position
{
    Position(int x, int y, Orientation orientation);

    bool operator==(const Position&) const;    
};

#endif
\end{c++}
\end{minipage}%
\hfill\vrule\hfill
\begin{minipage}[t]{0.45\linewidth}
\begin{c++}[caption={include/robot-cleaner/Position.h}]
#ifndef H722C49C6_D285_4885_88D1_97A11D669EE1
#define H722C49C6_D285_4885_88D1_97A11D669EE1

#include "robot-cleaner/Orientation.h"

struct Position
{
    Position(int x, int y, Orientation orientation);

    bool operator==(const Position&) const;
    
private:
    int x;
    int y;
    Orientation orientation;
};

#endif
\end{c++}
\end{minipage}%

实现比较逻辑也较为简单,创建\ascii{Position.cpp}直接实现。

\begin{leftbar}
\begin{c++}[caption={src/robot-cleaner/Position.cpp}]
#include "robot-cleaner/Position.h"

Position::Position(int x, int y, Orientation orientation)
  : x(x), y(y), orientation(orientation)
{}

bool Position::operator==(const Position& rhs) const
{
    return x == rhs.x 
        && y == rhs.y
        && orientation = rhs.orientation;
}
\end{c++}
\end{leftbar}

\subsection{重构}

\subsubsection{显式实现}

首先消除\ascii{RobotCleaner}的伪实现,创建一个\ascii{Position}的成员变量,并将\ascii{getPosition}的返回值类型修正为\ascii{const}引用。

\begin{minipage}[t]{0.45\linewidth}
\begin{c++}[caption={include/robot-cleaner/RobotCleaner.h}]
#ifndef H632C2FEA_66F0_4B68_B047_121B6CE1482E
#define H632C2FEA_66F0_4B68_B047_121B6CE1482E

struct Position;

struct RobotCleaner
{
    Position getPosition() const;
};

#endif
\end{c++}
\end{minipage}%
\hfill\vrule\hfill
\begin{minipage}[t]{0.45\linewidth}
\begin{c++}[caption={include/robot-cleaner/RobotCleaner.h}]
#ifndef H632C2FEA_66F0_4B68_B047_121B6CE1482E
#define H632C2FEA_66F0_4B68_B047_121B6CE1482E

#include "robot-cleaner/Position.h"

struct RobotCleaner
{
    RobotCleaner();

    const Position& getPosition() const;
    
private:
    Position position;
};

#endif
\end{c++}
\end{minipage}%

最后并将初始化的代码移植到构造函数中去。

\begin{minipage}[t]{0.45\linewidth}
\begin{c++}[caption={src/robot-cleaner/RobotCleaner.cpp}]
#include "robot-cleaner/RobotCleaner.h"

Position RobotCleaner::getPosition() const
{
    return Position(0, 0, NORTH);
}
\end{c++}
\end{minipage}%
\hfill\vrule\hfill
\begin{minipage}[t]{0.45\linewidth}
\begin{c++}[caption={src/robot-cleaner/RobotCleaner.cpp}]
#include "robot-cleaner/RobotCleaner.h"

RobotCleaner::RobotCleaner()
  : position(0, 0, NORTH)
{}

const Position& RobotCleaner::getPosition() const
{
    return position;
}
\end{c++}
\end{minipage}%

\end{content}

\section{left指令}

\begin{content}

\subsection{测试用例}

\begin{leftbar}
\begin{c++}[caption={test/robot-cleaner/TestRobotCleaner.h}]   
TEST("the robot should be in (0, 0, W) when I send left instruction")
{
    robot.turnLeft();
    ASSERT_EQ(Position(0, 0, WEST), robot.getPosition());
}
\end{c++}
\end{leftbar}

\subsection{通过测试}

为了快速通过测试用例,为\ascii{RobotCleaner}新增一个\ascii{turnLeft}成员函数。

\begin{minipage}[t]{0.45\linewidth}
\begin{c++}[caption={include/robot-cleaner/RobotCleaner.h}]
#ifndef H632C2FEA_66F0_4B68_B047_121B6CE1482E
#define H632C2FEA_66F0_4B68_B047_121B6CE1482E

#include "robot-cleaner/Position.h"

struct RobotCleaner
{
    RobotCleaner();

    const Position& getPosition() const;
    
private:
    Position position;
};

#endif
\end{c++}
\end{minipage}%
\hfill\vrule\hfill
\begin{minipage}[t]{0.45\linewidth}
\begin{c++}[caption={include/robot-cleaner/RobotCleaner.h}]
#ifndef H632C2FEA_66F0_4B68_B047_121B6CE1482E
#define H632C2FEA_66F0_4B68_B047_121B6CE1482E

#include "robot-cleaner/Position.h"

struct RobotCleaner
{
    RobotCleaner();
    
    void turnLeft();

    const Position& getPosition() const;
    
private:
    Position position;
};

#endif
\end{c++}
\end{minipage}%

此处,对于实现\ascii{turnLeft}还没有更多的想法,先伪实现,确保测试尽快通过。

\begin{leftbar}
\begin{c++}[caption={src/robot-cleaner/RobotCleaner.cpp}]
void RobotCleaner::turnLeft()
{
    position = Position(0, 0, WEST);
}
\end{c++}
\end{leftbar}

\subsection{重构}

\subsubsection{显式实现}

先将\ascii{turnLeft}委托给\ascii{Position}。

\begin{leftbar}
\begin{c++}[caption={src/robot-cleaner/RobotCleaner.cpp}]
void RobotCleaner::turnLeft()
{
    position.turnLeft();
}
\end{c++}
\end{leftbar}

在\ascii{make}之前,先将\ascii{Position::turnLeft()}伪实现。

\begin{leftbar}
\begin{c++}[caption={src/robot-cleaner/Position.cpp}]
void Position::turnLeft()
{
    orientation = WEST;
}
\end{c++}
\end{leftbar}

确保测试通过后,继续消除\ascii{Position}的伪实现,为此再写一个测试,让问题变得更加明确。

\begin{leftbar}
\begin{c++}[caption={test/robot-cleaner/TestRobotCleaner.h}]
TEST("the robot should be in (0, 0, S) when I send left instruction with 2 times")
{
    robot.turnLeft();
    robot.turnLeft();
    ASSERT_EQ(Position(0, 0, SOUTH), robot.getPosition());
}
\end{c++}
\end{leftbar}

为了快速让测试通过,先使用\ascii{switch-case}快速实现。

\begin{tabular}{@{}p{\dimexpr0.45\linewidth-\tabcolsep} 
                 | p{\dimexpr0.45\linewidth-\tabcolsep}@{}}
\begin{c++}[caption={src/robot-cleaner/Position.cpp}]
void Position::turnLeft()
{
    orientation = WEST;
}
\end{c++}
&
\begin{c++}[caption={src/robot-cleaner/Position.cpp}]
namespace
{
    Orientation toLeft(Orientation orientation)
    {
        switch (orientation)
        {
        case WEST:  return SOUTH;
        case NORTH: return WEST;
        default: break;
        }     
        return orientation;
    }
}

void Position::turnLeft()
{
    orientation = toLeft(orientation);
}
\end{c++}
\end{tabular}

重复这个微小的循环,覆盖所有的逻辑,确保自己永不犯错,每次通过测试都是修改非常少的代码而完成的,即使犯错也是非常容易定位和修正。

\begin{leftbar}
\begin{c++}[caption={test/robot-cleaner/TestRobotCleaner.h}]
TEST("the robot should be in (0, 0, E) when I send left instruction with 3 times")
{
    robot.turnLeft();
    robot.turnLeft();
    robot.turnLeft();

    ASSERT_EQ(Position(0, 0, EAST), robot.getPosition());
}

TEST("the robot should be in (0, 0, N) when I send left instruction with 4 times")
{
    robot.turnLeft();
    robot.turnLeft();
    robot.turnLeft();
    robot.turnLeft();
    
    ASSERT_EQ(Position(0, 0, NORTH), robot.getPosition());
}
\end{c++}
\end{leftbar}

\ascii{toLeft}的实现也就被驱动完成了。

\begin{leftbar}
\begin{c++}[caption={src/robot-cleaner/Position.cpp}]
namespace
{
    Orientation toLeft(Orientation orientation)
    {
        switch (orientation)
        {
        case EAST:  return NORTH;
        case SOUTH: return EAST;
        case WEST:  return SOUTH;
        case NORTH: return WEST;
        default: break;
        }     
        return orientation;
    }
}

void Position::turnLeft()
{
    orientation = toLeft(orientation);
}
\end{c++}
\end{leftbar}

也许与你共鸣,这样的设计非常不理想,已经存在很多明显的坏味道了。

\begin{enum}
\eitem{\ascii{switch-case}语句也让设计变得僵化;}
\eitem{\ascii{Position}的\ascii{turnLeft}具有副作用;}
\eitem{测试用例存在明显的重复代码;}
\eitem{\ascii{turnLeft}具有严重的传染性,从\ascii{RobotCleaner}开始一直传递到\ascii{Position}。}
\end{enum}

但不管怎么样,我们依然还要继续,因为到目前为止,需求我们只完成了一半。此刻更需要拒绝诱惑,在测试通过之前就将设计做大做全,不仅仅损伤自信心,而且更让设计没完没了,走不到尽头。是的,让测试通过是压倒一切的中心任务。为了快速通过测试的过程之中所使用的一切不光彩的行为,要不了多长时间,我们会回来的。

\end{content}

\section{right指令}

\begin{content}

因为我们对于问题域已经基本明确,此外抛开设计良好与否,\ascii{turnRight}的实现基本上是通过\ascii{copy-paste}完成的。所以在\ascii{turnLeft}实现的经验之上,这次将步子稍微迈大一点,保证我们开发慢慢步入快速车道。总之需要根据实际情况,保持进度有快有慢,收放自如。

但无论怎么样,还是要坚持如上述实现\ascii{toLeft}的步骤,小步实现\ascii{toRight},这是一种良好的习惯。

\subsection{快速实现}

\begin{leftbar}
\begin{c++}[caption={test/robot-cleaner/TestRobotCleaner.h}]    
TEST("the robot should be in (0, 0, E) when I send right instruction")
{
    robot.turnRight();
    ASSERT_EQ(Position(0, 0, EAST), robot.getPosition());
}

TEST("the robot should be in (0, 0, S) when I send right instruction with 2-times")
{
    robot.turnRight();
    robot.turnRight();
    
    ASSERT_EQ(Position(0, 0, SOUTH), robot.getPosition());
}

TEST("the robot should be in (0, 0, W) when I send right instruction with 3-times")
{
    robot.turnRight();
    robot.turnRight();
    robot.turnRight();

    ASSERT_EQ(Position(0, 0, WEST), robot.getPosition());
}

TEST("the robot should be in (0, 0, N) when I send right instruction with 4-times")
{
    robot.turnRight();
    robot.turnRight();
    robot.turnRight();
    robot.turnRight();

    ASSERT_EQ(Position(0, 0, NORTH), robot.getPosition());
}
\end{c++}
\end{leftbar}

\ascii{turnRight}也非常简单,直接显示地实现。

\begin{leftbar}
\begin{c++}[caption={src/robot-cleaner/RobotCleaner.cpp}]
void RobotCleaner::turnRight()
{
    position.turnRight();
}
\end{c++}
\end{leftbar}

\begin{leftbar}
\begin{c++}[caption={src/robot-cleaner/Position.cpp}]
namespace
{
    Orientation toRight(Orientation orientation)
    {
        switch (orientation)
        {
        case EAST:  return SOUTH;
        case SOUTH: return WEST;
        case WEST:  return NORTH;
        case NORTH: return EAST;
        default: break;
        }
        return orientation;
    }
}

void Position::turnRight()
{
    orientation = toRight(orientation);
}
\end{c++}
\end{leftbar}

到目前为止,设计已经充满着坏味道,已经令人窒息。

\begin{enum}
\eitem{\ascii{turnLeft}与\ascii{turnRight}存在重复代码,并且犹如病毒一样,并在调用链上传递,从而使每一类都不能封闭修改;}
\eitem{\ascii{Position}的\ascii{turnLeft},\ascii{turnRight}具有副作用;}
\eitem{测试代码也存在明显的重复设计;}
\eitem{两个\ascii{switch-case}进一步导致设计的僵化性。}
\end{enum}

\end{content}

\section{改善设计}

\begin{content}
     在这之前所有的实现,都是为了快速地实现需求,即使重构也做得非常小,每次改动都能保证测试的通过。但在实现需求的路上,也欠下了很多技术债务,为此在测试保护的情况下需要进一步改善即有的设计。

\subsection{消除重复}

当实现\ascii{turnRight}时,已经释放出强烈的坏味道,\ascii{turnLeft}与\ascii{turnRight}高度重复。消除\ascii{turnLeft}与\ascii{turnRight}之间的重复迫在眉睫。

先改进\ascii{Position}的设计,使用\ascii{Position::turnTo}合并实现\ascii{Position::turnLeft}与\ascii{Position::turnRight}的实现。

\begin{tabular}{@{}p{\dimexpr0.45\linewidth-\tabcolsep}% can be up to 0.5\linewidth 
                 | p{\dimexpr0.45\linewidth-\tabcolsep}@{}}
\begin{c++}[caption={src/robot-cleaner/Position.cpp}]
namespace
{
    Orientation toLeft(Orientation orientation)
    {
        switch (orientation)
        {
        case EAST:  return NORTH;
        case SOUTH: return EAST;
        case WEST:  return SOUTH;
        case NORTH: return WEST;
        default: break;
        }   
        return orientation;
    }

    Orientation toRight(Orientation orientation)
    {
        switch (orientation)
        {
        case EAST:  return SOUTH;
        case SOUTH: return WEST;
        case WEST:  return NORTH;
        case NORTH: return EAST;
        default: break;
        }
        return orientation;
    }
}

void Position::turnLeft()
{
    orientation = toLeft(orientation);
}

void Position::turnRight()
{
    orientation = toRight(orientation);
}    
\end{c++}
&
\begin{c++}[caption={src/robot-cleaner/Position.cpp}]
namespace
{
    Orientation toLeft(Orientation orientation)
    {
        switch (orientation)
        {
        case EAST:  return NORTH;
        case SOUTH: return EAST;
        case WEST:  return SOUTH;
        case NORTH: return WEST;
        default: break;
        }
        
        return orientation;
    }

    Orientation toRight(Orientation orientation)
    {
        switch (orientation)
        {
        case EAST:  return SOUTH;
        case SOUTH: return WEST;
        case WEST:  return NORTH;
        case NORTH: return EAST;
        default: break;
        }
        return orientation;
    }
    
    Orientation to(bool left, Orientation orientation)
    {
       return left ? toLeft(orientation) : toRight(orientation);
    }
}

void Position::turnTo(bool left)
{
    orientation = to(left, orientation);
}
\end{c++}
\end{tabular}

修改原来对\ascii{Position::turnLeft}与\ascii{Position::turnRight}的所有调用点,保证测试运行通过。

\begin{tabular}{@{}p{\dimexpr0.45\linewidth-\tabcolsep} 
                 | p{\dimexpr0.45\linewidth-\tabcolsep}@{}}
\begin{c++}[caption={src/robot-cleaner/RobotCleaner.cpp}]
void RobotCleaner::turnLeft()
{
    position.turnLeft();
}

void RobotCleaner::turnRight()
{
    position.turnRight();
}
\end{c++}
&
\begin{c++}[caption={src/robot-cleaner/RobotCleaner.cpp}]
void RobotCleaner::turnLeft()
{
    position.turnTo(true);
}

void RobotCleaner::turnRight()
{
    position.turnTo(false);
}
\end{c++}
\end{tabular}

虽然有晦涩的\ascii{true/false}接口,但暂时不要考虑,继续将\ascii{RobotCleaner::turnLeft}与\ascii{RobotCleaner::turnRight}之间的重复彻底消除。

\begin{tabular}{@{}p{\dimexpr0.45\linewidth-\tabcolsep} 
                 | p{\dimexpr0.45\linewidth-\tabcolsep}@{}}
\begin{c++}[caption={src/robot-cleaner/RobotCleaner.cpp}]
void RobotCleaner::turnLeft()
{
    position.turnTo(true);
}

void RobotCleaner::turnRight()
{
    position.turnTo(false);
}
\end{c++}
&
\begin{c++}[caption={src/robot-cleaner/RobotCleaner.cpp}]
void RobotCleaner::turnTo(bool left)
{
    position.turnTo(left);
}
\end{c++}
\end{tabular}

修改原来对\ascii{RobotCleaner::turnLeft}与\ascii{RobotCleaner::turnRight}的所有调用点,确保测试快速通过。

\begin{tabular}{@{}p{\dimexpr0.45\linewidth-\tabcolsep} 
                 | p{\dimexpr0.45\linewidth-\tabcolsep}@{}}
\begin{c++}[caption={test/robot-cleaner/TestRobotCleaner.h}]
FIXTURE(RobotCleaner)
{
    RobotCleaner robot;

    TEST("the robot should be in (0, 0, W) when I send instruction left")
    {
        robot.turnLeft();
        ASSERT_EQ(Position(0, 0, WEST), robot.getPosition());
    }

    TEST("the robot should be in (0, 0, E) when I send instruction right")
    {
        robot.turnRight();
        ASSERT_EQ(Position(0, 0, EAST), robot.getPosition());
    }
};
\end{c++}
&
\begin{c++}[caption={test/robot-cleaner/TestRobotCleaner.h}]
FIXTURE(RobotCleaner)
{
    RobotCleaner robot;

    TEST("the robot should be in (0, 0, W) when I send instruction left")
    {
        robot.turnTo(true);
        ASSERT_EQ(Position(0, 0, WEST), robot.getPosition());
    }

    TEST("the robot should be in (0, 0, E) when I send instruction right")
    {
        robot.turnTo(false);
        ASSERT_EQ(Position(0, 0, EAST), robot.getPosition());
    }
};
\end{c++}
\end{tabular}

晦涩的\ascii{true/false}用户接口表现非常不佳,引来了很多反对的声音。有人建议将接口改回去,因为他们受不了这样的用户接口。

事实上,消除\ascii{turnLeft}和\ascii{turnRight}重复优先级要更高,晦涩的\ascii{true/false}用户接口的表达力可以等待进一步改善和提高。

此刻的重构只是一个开始,重构可是一个循序渐进的过程,我们无法在一个时刻解决所有的问题。不要担心目前设计的存在的一些问题,关键在于我们是否有持续演进的勇气和决心。

\subsection{改善表达力}

\subsubsection{改善bool接口的调用}

首先要识别出\ascii{bool}接口到底存在什么样的坏味道?当用户调用\asccii{true}或者\asccii{false}的时候,不知道是什么含义,尤其调用点和被调用的距离越远,则问题越发突出。

但是,当传递\ascii{bool}变量,而非字面值\ascii{true/false}的时候,如果变量的名字具有明显\ascii{bool}语义,例如携带\ascii{is, has, should}等前缀,则不存在接口调用时晦涩的问题。此刻\ascii{bool}变量就是一个普通的变量而已,与其他变量传递没有其他特别之处。

所以不要盲目地排斥\ascii{bool}接口,有时候合理运用,则能消除大量的重复代码。对于目前的问题,首先需要改善晦涩的\ascii{true/false}的调用点,提高表达力。

\begin{tabular}{@{}p{\dimexpr0.45\linewidth-\tabcolsep} 
                 | p{\dimexpr0.45\linewidth-\tabcolsep}@{}}
\begin{c++}[caption={test/robot-cleaner/TestRobotCleaner.h}]
FIXTURE(RobotCleaner)
{
    RobotCleaner robot;

    TEST("the robot should be in (0, 0, W) when I send instruction left")
    {
        robot.turnTo(true);
        ASSERT_EQ(Position(0, 0, WEST), robot.getPosition());
    }

    TEST("the robot should be in (0, 0, E) when I send instruction right")
    {
        robot.turnTo(false);
        ASSERT_EQ(Position(0, 0, EAST), robot.getPosition());
    }
};
\end{c++}
&                 
\begin{c++}[caption={test/robot-cleaner/TestRobotCleaner.h}]
FIXTURE(RobotCleaner)
{
    RobotCleaner robot;

    TEST("the robot should be in (0, 0, W) when I send instruction left")
    {
        robot.exec(LEFT);
        ASSERT_EQ(Position(0, 0, WEST), robot.getPosition());
    }

    TEST("the robot should be in (0, 0, E) when I send instruction right")
    {
        robot.exec(RIGHT);
        ASSERT_EQ(Position(0, 0, EAST), robot.getPosition());
    }
};
\end{c++}
\end{tabular}

为此新建\ascii{Instruction.h}的头文件,用于表示\ascii{RobotCleaner}使用的指令集。

\begin{leftbar}
\begin{c++}[caption={include/robot-cleaner/Instruction.h}]
#ifndef HC37C3D94_43F1_4677_BD56_34CB78EFEC75
#define HC37C3D94_43F1_4677_BD56_34CB78EFEC75

enum Instruction { LEFT, RIGHT };

#endif
\end{c++}
\end{leftbar}

修改\ascii{RobotCleaner}的接口设计和实现。

\begin{tabular}{@{}p{\dimexpr0.45\linewidth-\tabcolsep} 
                 | p{\dimexpr0.45\linewidth-\tabcolsep}@{}}
\begin{c++}[caption={include/robot-cleaner/RobotCleaner.h}]
void RobotCleaner::turnTo(bool left)
{
    position.turnTo(left);
}
\end{c++}
&                 
\begin{c++}[caption={include/robot-cleaner/RobotCleaner.h}]
void RobotCleaner::exec(Instruction instruction)
{
    position.turnTo(instruction == LEFT);
}
\end{c++}
\end{tabular}

\ascii{true/false}字面值的之间调用,被替换为\ascii{instruction == LEFT}的调用,在局部上改进了用户的表达力。但到目前位置,我所力所能及了。至于是否存在更好的解决方案,等测试更加充分,需求变得更加明确时,在进一步改善。

\subsubsection{改善用例的表达力}

使用\ascii{BDD}改善用例的可读性。不仅如此,\ascii{BDD}让你更好地实践\ascii{TDD}。

\begin{tabular}{@{}p{\dimexpr0.45\linewidth-\tabcolsep} 
                 | p{\dimexpr0.45\linewidth-\tabcolsep}@{}}
\begin{c++}[caption={test/robot-cleaner/TestRobotCleaner.h}]
FIXTURE(RobotCleaner)
{
    RobotCleaner robot;
    
    TEST("at the beginning, the robot should be in at the initial position")
    {
        ASSERT_EQ(Position(0, 0, NORTH), robot.getPosition());
    }

    TEST("the robot should be in (0, 0, W) when I send instruction left")
    {
        robot.exec(LEFT);
        ASSERT_EQ(Position(0, 0, WEST), robot.getPosition());
    }

    TEST("the robot should be in (0, 0, E) when I send instruction right")
    {
        robot.exec(RIGHT);
        ASSERT_EQ(Position(0, 0, EAST), robot.getPosition());
    }
};
\end{c++}
&
\begin{c++}[caption={test/robot-cleaner/TestRobotCleaner.h}]
FIXTURE(RobotCleaner)
{
    RobotCleaner robot;
    
    void WHEN_I_send_instruction(Instruction instruction)
    {
        robot.exec(instruction);
    }
    
    void AND_I_send_instruction(Instruction instruction)
    {
        robot.exec(instruction);
    }    

    void THEN_the_robot_cleaner_should_be_in(const Position& position)
    {
        ASSERT_EQ(position, robot.getPosition());
    }
    
    TEST("at the beginning")
    {
        THEN_the_robot_cleaner_should_be_in(Position(0, 0, NORTH));
    }

    TEST("left instruction")
    {
        WHEN_I_send_instruction(LEFT);
        THEN_the_robot_cleaner_should_be_in(Position(0, 0, WEST));
    }

    TEST("right instruction")
    {
        WHEN_I_send_instruction(RIGHT);
        THEN_the_robot_cleaner_should_be_in(Position(0, 0, EAST));
    }
};
\end{c++}
\end{tabular}

\subsection{消除条件分支语句}

快速实现\ascii{Position}时,遗留两个\ascii{switch-case},让设计瞬间失去了光芒。透过对“方向”概念本质的理解,对于方向\ascii{turnTo}的操作,可以使用表驱动,即可消除糟糕的\ascii{switch-case}实现。

\begin{tabular}{@{}p{\dimexpr0.45\linewidth-\tabcolsep} 
                 | p{\dimexpr0.45\linewidth-\tabcolsep}@{}}
\begin{c++}[caption={src/robot-cleaner/Position.cpp}]
namespace
{
    Orientation toLeft(Orientation orientation)
    {
        switch (orientation)
        {
        case EAST:  return NORTH;
        case SOUTH: return EAST;
        case WEST:  return SOUTH;
        case NORTH: return WEST;
        default: break;
        }
        
        return orientation;
    }

    Orientation toRight(Orientation orientation)
    {
        switch (orientation)
        {
        case EAST:  return SOUTH;
        case SOUTH: return WEST;
        case WEST:  return NORTH;
        case NORTH: return EAST;
        default: break;
        }
        return orientation;
    }
    
    Orientation to(bool left, Orientation orientation)
    {
       return left ? toLeft(orientation) : toRight(orientation);
    }
}

void Position::turnTo(bool left)
{
    orientation = to(left, orientation);
}
\end{c++}
&
\begin{c++}[caption={src/robot-cleaner/Position.cpp}]
namespace
{
    Orientation orientations[] = { EAST, SOUTH, WEST, NORTH };

    inline Orientation toLeft(Orientation orientation)
    {
        return orientations[(orientation + 3) % 4];
    }
    
    inline Orientation toRight(Orientation orientation)
    {
        return orientations[(orientation + 1) % 4];
    }

    inline Orientation to(bool left, Orientation orientation)
    {
        return left ? toLeft(orientation) : toRight(orientation);
    }
}

void Position::turnTo(bool left)
{
    orientation = to(left, orientation);
}
\end{c++}
\end{tabular}

\ascii{toLeft}与\ascii{toRight}算法实现依然存在重复,进一步消除重复。

\begin{tabular}{@{}p{\dimexpr0.45\linewidth-\tabcolsep} 
                 | p{\dimexpr0.45\linewidth-\tabcolsep}@{}}
\begin{c++}[caption={src/robot-cleaner/Position.cpp}]
namespace
{
    Orientation orientations[] = { EAST, SOUTH, WEST, NORTH };

    inline Orientation toLeft(Orientation orientation)
    {
        return orientations[(orientation + 3) % 4];
    }
    
    inline Orientation toRight(Orientation orientation)
    {
        return orientations[(orientation + 1) % 4];
    }

    inline Orientation to(bool left, Orientation orientation)
    {
        return left ? toLeft(orientation) : toRight(orientation);
    }
}

void Position::turnTo(bool left)
{
    orientation = to(left, orientation);
}
\end{c++}
&
\begin{c++}[caption={src/robot-cleaner/Position.cpp}]
namespace
{
    Orientation orientations[] = { EAST, SOUTH, WEST, NORTH };

    inline int numOfTurnTo(bool left)
    { return left ? 3 : 1; }
}

void Position::turnTo(bool left)
{
    orientation = orientations[(orientation + numOfTurnTo(left)) % 4];
}
\end{c++}
\end{tabular}
 
重构完毕,测试通过,设计也变得较为清晰。
    
\end{content}

\section{遗留问题}

\begin{content}

至此,\ascii{TurnTo}已经实现了,但存在一些明显的坏味道,将其记录在\ascii{to-do list}中,后续再做改善。

\begin{enum}
\eitem{\ascii{Position}是一个典型的\ascii{Value Object},\ascii{turnTo}的实现具有副作用;}
\eitem{\ascii{Orientation}的枚举成员顺序与\ascii{Position.cpp}中定义的\ascii{orientations}数组元素的顺序,它们都是“方向顺序”这一知识的两种不同表现形式,具有设计的重复性;更为严重的是\ascii{turnTo}的算法实现强依赖于这样的约定,设计具有脆弱性;}
\eitem{\ascii{RobotCleaner::exec}过于僵硬,扩展支持实现其他的指令时,问题更为突出;}
\eitem{\ascii{turnTo}的传染性,让设计变得更加耦合;}
\eitem{测试具有重复性。}
\end{enum}

\end{content}
