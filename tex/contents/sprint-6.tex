\begin{savequote}[45mm]
\ascii{Controlling complexity is the essence of computer programming.} 
\qauthor{\ascii{- Brian Kernighan}}
\end{savequote}

\chapter{Repeat} 
\label{ch:repeat}

\section{需求}

\begin{content}

当\ascii{Robot}收到\ascii{REAPT(instruction, n)}指令后,它会循环执行\ascii{instruction}指令\ascii{n}次。原来在\ascii{(0, 0, N)},执行\ascii{REAPT(FORWARD(2), 2)}指令后,新的位置为\ascii{(0, 4, NORTH)};其中\ascii{n}在\ascii{[1..10]}之间。

\end{content}

\section{repeat指令}

\begin{content}

\subsection{测试用例}

\begin{leftbar}
\begin{c++}[caption={test/robot-cleaner/TestRobotCleaner.h}]
TEST("repeat instruction")
{
    WHEN_I_send_instruction(repeat(forward_n(2), 2));
    THEN_the_robot_cleaner_should_be_in(Position(0, 4, NORTH));
}
\end{c++}
\end{leftbar}

\subsection{实现repeat指令}

提供\ascii{repeat}关键字,并将\ascii{round}指令内联。

\begin{tabular}{@{}p{\dimexpr0.45\linewidth-\tabcolsep} 
                 | p{\dimexpr0.45\linewidth-\tabcolsep}@{}}
\begin{c++}[caption={include/robot-cleaner/Instruction.h}]
#ifndef HC37C3D94_43F1_4677_BD56_34CB78EFEC75
#define HC37C3D94_43F1_4677_BD56_34CB78EFEC75

#include "base/Role.h"

struct Position;

DEFINE_ROLE(Instruction)
{
    ABSTRACT(void exec(Position& position));
};

Instruction* repeat(Instruction* instruction, int n);

inline Instruction* round()
{ return repeat(right(), 2); }
\end{c++}
&
\begin{c++}[caption={src/robot-cleaner/Instruction.cpp}]
namespace
{
    struct RepeatedInstruction : Instruction
    {
        RepeatedInstruction(Instruction* instruction, int n)
         : instruction(instruction), n(n)
        {}

    private:
        OVERRIDE(void exec(Point& point, Orientation& orientation) const)
        {
            for (auto i=0; i<n; i++)
            {
                instruction->exec(point, orientation);
            }
        }

    private:
        std::unique_ptr<Instruction> instruction;
        int n;
    };
}

Instruction* repeat(Instruction* instruction, int n)
{
    return new RepeatedInstruction(instruction, n);
}
\end{c++}
\end{tabular}

测试通过,功能实现了。

\subsection{边界}

\subsubsection{测试用例}

\begin{leftbar}
\begin{c++}[caption={test/robot-cleaner/TestRobotCleaner.h}]
TEST("repeat instruction: n out of bound")
{
    WHEN_I_send_instruction(repeat(forward_n(2), 11));
    THEN_the_robot_cleaner_should_be_in(Position(0, 0, NORTH));
}
\end{c++}
\end{leftbar}

\subsubsection{快速实现}

\begin{tabular}{@{}p{\dimexpr0.45\linewidth-\tabcolsep} 
                 | p{\dimexpr0.45\linewidth-\tabcolsep}@{}}
\begin{c++}[caption={src/robot-cleaner/Instruction.cpp}]
namespace
{
    struct RepeatedInstruction : Instruction
    {
        RepeatedInstruction(Instruction* instruction, int n)
         : instruction(instruction), n(n)
        {}

    private:
        OVERRIDE(void exec(Point& point, Orientation& orientation) const)
        {
            for (auto i=0; i<n; i++)
            {
                instruction->exec(point, orientation);
            }
        }

    private:
        std::unique_ptr<Instruction> instruction;
        int n;
    };
}

Instruction* repeat(Instruction* instruction, int n)
{
    return new RepeatedInstruction(instruction, n);
}
\end{c++}
&
\begin{c++}[caption={src/robot-cleaner/Instruction.cpp}]
namespace
{
    const int MIN_REPEATED_NUM = 1;
    const int MAX_REPEATED_NUM = 10;

    struct RepeatedInstruction : Instruction
    {
        RepeatedInstruction(Instruction* instruction, int n)
         : instruction(instruction), n(n)
        {}

    private:
        OVERRIDE(void exec(Point& point, Orientation& orientation) const)
        {
            for (auto i=0; i<n; i++)
            {
                instruction->exec(point, orientation);
            }
        }
       
    private:
        static int protect(int n)
        {
            return (MIN_REPEATED_NUM <= n
              && n <= MAX_REPEATED_NUM) ? n : 0;
        }

    private:
        std::unique_ptr<Instruction> instruction;
        int n;
    };
}

Instruction* repeat(Instruction* instruction, int n)
{
    return new RepeatedInstruction(instruction, n);
}
\end{c++}
\end{tabular}

测试通过了,但实现的很不光彩,而且与\ascii{MoveOnInstruction}的边界处理产生了重复设计,不仅仅是实现上的重复,关键在与对越界处理的知识的重复(越界时,规整为\ascii{0}),需要进一步消除重复。

\subsubsection{Null Object}

为了统一\ascii{MoveToInstruction}与\ascii{RepeatedInstruction}的边界处理,此处可引入\ascii{Null Object}模式,将边界处理得更加自然。

\begin{tabular}{@{}p{\dimexpr0.45\linewidth-\tabcolsep} 
                 | p{\dimexpr0.45\linewidth-\tabcolsep}@{}}
\begin{c++}[caption={src/robot-cleaner/Instruction.cpp}]
namespace
{
    const int MIN_MOVE_NUM = 1;
    const int MAX_MOVE_NUM = 10;
    
    struct MoveOnInstruction : Instruction
    {
        explicit MoveOnInstruction(int n, bool forward)
          : step(prefix(forward)*protect(n)) {}

    private:
        OVERRIDE(void exec(Position& position))
        { 
            position.moveOn(step); 
        }

    private:
        static int prefix(bool forward)
        { 
            return forward ? 1 : -1; 
        }
        
        static int protect(int n)
        {
            return (MIN_MOVE_NUM <= n 
              && n <= MAX_MOVE_NUM) ? n : 0;
        }

    private:
        int step;
    };
}  
  
Instruction* forward_n(int n)
{ 
    return new MoveOnInstruction(n, true); 
}

Instruction* backward_n(int n)
{ 
    return new MoveOnInstruction(n, false); 
}

namespace
{
    const int MIN_REPEATED_NUM = 1;
    const int MAX_REPEATED_NUM = 10;

    struct RepeatedInstruction : Instruction
    {
        RepeatedInstruction(Instruction* instruction, int n)
         : instruction(instruction), n(protect(n))
        {}
        
    private:
        OVERRIDE(void exec(Position& position))
        {
            for (auto i=0; i<n; i++)
            {
                instruction->exec(position);
            }
        }
        
    private:
        static int protect(int n)
        {
            return (MIN_REPEATED_NUM <= n 
              && n <= MAX_REPEATED_NUM) ? n : 0;
        }

    private:
        std::unique_ptr<Instruction> instruction;
        int n;
    };
}

Instruction* repeat(Instruction* instruction, int n)
{ return new RepeatedInstruction(instruction, n); }
\end{c++}
&
\begin{c++}[caption={src/robot-cleaner/Instruction.cpp}]
namespace
{
#define ASSERT_BETWEEN(num, min, max)   \
    do {                                \
       if (num<min || num>max)          \
          return new EmptyInstruction   \
    } while(0)

    struct EmptyInstruction : Instruction
    { 
        OVERRIDE(void exec(Position&)) {} 
    };
}

namespace
{
    struct MoveOnInstruction : Instruction
    {
        explicit MoveOnInstruction(int n, bool forward)
          : step(prefix(forward)*n) {}

    private:
        OVERRIDE(void exec(Position& position))
        { 
            position.moveOn(step); 
        }

    private:
        static int prefix(bool forward)
        { 
            return forward ? 1 : -1; 
        }
        
    private:
        int step;
    };
    
    const int MIN_MOVE_NUM = 1;
    const int MAX_MOVE_NUM = 10;

    Instruction* move_on(bool forward, int n)
    {
        ASSERT_BETWEEN(n, MIN_MOVE_NUM, MAX_MOVE_NUM);
        return new MoveOnInstruction(forward,  n);
    }
}

Instruction* forward_n(int n)
{ 
    return move_on(true, n); 
}

Instruction* backward_n(int n)
{ 
    return move_on(true, n); 
}

namespace
{
    struct RepeatedInstruction : Instruction
    {
        RepeatedInstruction(Instruction* instruction, int n)
         : instruction(instruction), n(n)
        {}
        
    private:
        OVERRIDE(void exec(Position& position))
        {
            for (auto i=0; i<n; i++)
            {
                instruction->exec(position);
            }
        }

    private:
        std::unique_ptr<Instruction> instruction;
        int n;
    };
    
    const int MIN_REPEATED_NUM = 1;
    const int MAX_REPEATED_NUM = 10;
}

Instruction* repeat(Instruction* instruction, int n)
{
    ASSERT_BETWEEN(n, MIN_REPEATED_NUM, MAX_REPEATED_NUM);
    return new RepeatedInstruction(instruction, n);
}
\end{c++}
\end{tabular}

测试通过了,重构相当成功。

\subsection{消除测试用例的重复}

用例之间存在重复代码,可以进一步进行提取和封装,消除重复代码。

\begin{tabular}{@{}p{\dimexpr0.45\linewidth-\tabcolsep} 
                 | p{\dimexpr0.45\linewidth-\tabcolsep}@{}}
\begin{c++}[caption={test/robot-cleaner/TestRobotCleaner.h}]
FIXTURE(RobotCleaner)
{
    RobotCleaner robot;
    
    TEST("left instruction: 1-times")
    {
        WHEN_I_send_instruction(left());
        THEN_the_robot_cleaner_should_be_in(Position(0, 0, WEST));
    }
    
    TEST("left instruction: 2-times")
    {
        WHEN_I_send_instruction(left());
        AND_I_send_instruction(left());

        THEN_the_robot_cleaner_should_be_in(Position(0, 0, SOUTH));
    }
    
    TEST("left instruction: 3-times")
    {
        WHEN_I_send_instruction(left());
        AND_I_send_instruction(left());
        AND_I_send_instruction(left());

        THEN_the_robot_cleaner_should_be_in(Position(0, 0, EAST));
    }
    
    TEST("left instruction: 4-times")
    {
        WHEN_I_send_instruction(left());
        AND_I_send_instruction(left());
        AND_I_send_instruction(left());
        AND_I_send_instruction(left());
        
        THEN_the_robot_cleaner_should_be_in(Position(0, 0, NORTH));
    }
};
\end{c++}
&
\begin{c++}[caption={test/robot-cleaner/TestRobotCleaner.h}]
FIXTURE(RobotCleaner)
{
    RobotCleaner robot;
        
    TEST("left instruction: 1-times")
    {
        WHEN_I_send_instruction(left());
        THEN_the_robot_cleaner_should_be_in(Position(0, 0, WEST));
    }
    
    TEST("left instruction: 2-times")
    {
        WHEN_I_send_instruction(repeat(left(), 2));
        THEN_the_robot_cleaner_should_be_in(Position(0, 0, SOUTH));
    }
    
    TEST("left instruction: 3-times")
    {
        WHEN_I_send_instruction(repeat(left(), 3));
        THEN_the_robot_cleaner_should_be_in(Position(0, 0, EAST));
    }
    
    TEST("left instruction: 4-times")
    {
        WHEN_I_send_instruction(repeat(left(), 4));
        THEN_the_robot_cleaner_should_be_in(Position(0, 0, NORTH));
    }
};
\end{c++}
\end{tabular}

测试通过,重构完毕。

\end{content}
