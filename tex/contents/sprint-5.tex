\begin{savequote}[45mm]  
\ascii{Make it work, make it right, make it fast.}
\qauthor{\ascii{- Kent Beck}}
\end{savequote}

\chapter{Sequential} 
\label{ch:sequential}

\section{需求}

\begin{content}

当\ascii{Robot}收到一系列组合指令时,能够依次按指令完成相应的动作。例如收到指令序列:\ascii{[LEFT, FORWARD, RIGHT, BACKWARD, ROUND, FORWARD(2)]},将依次执行:向左转\ascii{90}度;保持方向并向前移动一个坐标;向右转\ascii{90}度;保持方向并向后退一个坐标;顺时针旋转\ascii{180}度掉头;保持方向向前移动\ascii{2}个坐标。

\end{content}

\section{sequential指令}

\begin{content}

\subsection{测试用例}

\begin{leftbar}
\begin{c++}[caption={test/robot-cleaner/TestRobotCleaner.h}]
TEST("sequential instruction")
{
    WHEN_I_send_instruction(sequential(left(), forward(), round()));
    THEN_the_robot_cleaner_should_be_in(Position(-1, 0, EAST));
}
\end{c++}
\end{leftbar}

\subsection{实现sequential指令}

\begin{leftbar}
\begin{c++}[caption={include/robot-cleaner/Instruction.h}]
#ifndef HC37C3D94_43F1_4677_BD56_34CB78EFEC75
#define HC37C3D94_43F1_4677_BD56_34CB78EFEC75

#include "base/Role.h"
#include <initializer_list>

struct Position;

DEFINE_ROLE(Instruction)
{
    ABSTRACT(void exec(Position& position));
};

Instruction* __sequential(std::initializer_list<Instruction*>);

#define sequential(...) __sequential({ __VA_ARGS__ })

#endif
\end{c++}
\end{leftbar}

此处使用了\ascii{C++11}的特性,从而是接口变得更加直观、漂亮。

\begin{leftbar}
\begin{c++}[caption={src/robot-cleaner/Instruction.cpp}]
#include "robot-cleaner/Instruction.h"
#include <vector>
  
namespace
{
    struct SequentialInstruction : Instruction
    {
        SequentialInstruction(std::initializer_list<Instruction*> instructions)
          : instructions(instructions)
        {}

        ~SequentialInstruction()
        {
            for(auto instruction : instructions)
            {
                delete instruction;
            }
        }

    private:
        OVERRIDE(void exec(Position& position))
        {
            for(auto instruction : instructions)
            {
                instruction->exec(position);
            }
        }

    private:
        std::vector<Instruction*> instructions;
    };
}

Instruction* __sequential(std::initializer_list<Instruction*> instructions)
{
    return new SequentialInstruction(instructions);
}
\end{c++}
\end{leftbar}

\ascii{SequentialInstruction}采用了组合模式,整合了“一多”之分,在此使用也非常恰当。

\end{content}
