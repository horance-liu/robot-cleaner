\begin{savequote}[45mm]  
\ascii{Any fool can write code that a computer can understand. Good programmers write code that humans can understand.}
\qauthor{\ascii{- Martin Flower}}
\end{savequote}

\chapter{Orientation} 
\label{ch:orientation}

\section{改善设计}

\begin{content}

\subsection{重复设计}

至此,所有的需求和测试均以通过,但还存在一个重要的问题还未解决。关于\ascii{Orientation}枚举成员顺序的知识重复。目前已经存在\ascii{3}个地方暗喻了这个知识的存在,而且更为严重的是,\ascii{MoveOn}和\ascii{TurnTo}的算法都与它们紧密关联。

\begin{leftbar}
\begin{c++}[caption={include/robot-cleaner/Orientation.h}]
#ifndef H9811B75A_15B3_4DF0_91B7_483C42F74473
#define H9811B75A_15B3_4DF0_91B7_483C42F74473

enum Orientation { EAST, SOUTH, WEST, NORTH };

#endif
\end{c++}
\end{leftbar}

\begin{leftbar}
\begin{c++}[caption={src/robot-cleaner/Position.cpp}]
namespace
{
    Orientation orientations[] = { EAST, SOUTH, WEST, NORTH };
    const int offsets[] = { 1, 0, -1, 0 };
}
\end{c++}
\end{leftbar}

\subsection{安全枚举类型}

首先将\ascii{Orientation}从枚举提升为类,并快速恢复测试。

\begin{tabular}{@{}p{\dimexpr0.45\linewidth-\tabcolsep} 
                 | p{\dimexpr0.45\linewidth-\tabcolsep}@{}}
\begin{c++}[caption={include/robot-cleaner/Orientation.h}]
#ifndef H9811B75A_15B3_4DF0_91B7_483C42F74473
#define H9811B75A_15B3_4DF0_91B7_483C42F74473

enum Orientation { EAST, SOUTH, WEST, NORTH };

#endif
\end{c++}
&
\begin{c++}[caption={include/robot-cleaner/Orientation.h}]
#ifndef H9811B75A_15B3_4DF0_91B7_483C42F74473
#define H9811B75A_15B3_4DF0_91B7_483C42F74473

#include "base/EqHelper.h"

struct Orientation
{
    operator int() const;

#define DECL_ORIENTATION(orientation) \
    static const Orientation orientation;

    DECL_ORIENTATION(east)
    DECL_ORIENTATION(south)
    DECL_ORIENTATION(west)
    DECL_ORIENTATION(north)
    
    DECL_EQUALS(Point);

private:
    Orientation(int order);

private:
    int order;
};

#define EAST  Orientation::east
#define SOUTH Orientation::south
#define WEST  Orientation::west
#define NORTH Orientation::north

#endif
\end{c++}
\end{tabular}

此处暂时定义一个\ascii{Orientation::operator int},让\ascii{Orientation}工作起来,就犹如普通的枚举类型一样参与整数运算,保证测试用例尽快运行成功。

此时编译通过了,但链接不过,快速实现\ascii{Orientation.cpp}。

\begin{leftbar}
\begin{c++}[caption={src/robot-cleaner/Orientation.cpp}]
#include "robot-cleaner/Orientation.h"

Orientation::operator int() const
{ 
    return order; 
}

inline Orientation::Orientation(int order)
  : order(order)
{}

#define DEF_ORIENTATION(orientation, order) \
const Orientation Orientation::orientation(order);

DEF_ORIENTATION(east,  0)
DEF_ORIENTATION(south, 1)
DEF_ORIENTATION(west,  2)
DEF_ORIENTATION(north, 3)

DEF_EQUALS(Orientation)
{
    return FIELD_EQ(order);
}
\end{c++}
\end{leftbar}

测试通过了,此处\ascii{Orientation}维护了一个\ascii{order}变量,类似于普通枚举成员的值。

最后将原来\ascii{Orientation}的所有\ascii{pass-by-value}的地方都换成\ascii{pass-by-ref-to-const},保证测试运行通过。

\begin{tabular}{@{}p{\dimexpr0.45\linewidth-\tabcolsep} 
                 | p{\dimexpr0.45\linewidth-\tabcolsep}@{}}
\begin{c++}[caption={src/robot-cleaner/Position.cpp}]
Position::Position(int x, int y, Orientation orientation)
  : x(x), y(y), orientation(orientation)
{
}
\end{c++}
&
\begin{c++}[caption={src/robot-cleaner/Position.cpp}]
Position::Position(int x, int y, const Orientation& orientation)
  : x(x), y(y), orientation(orientation)
{
}
\end{c++}
\end{tabular}

如此设计,将\ascii{Orientation}枚举成员的顺序内聚在了一块,包括名称、编号。但它由原来单纯的枚举类型升级为类,将获得更大弹性的设计空间。

\subsection{搬迁职责}

将\ascii{Position}中关于\ascii{orientation}的一些逻辑搬迁至\ascii{Orientation}类中去,

\subsubsection{搬迁\ascii{TurnTo}}

将\ascii{TurnTo}的逻辑从\ascii{Position.cpp}搬迁过来,从而完全地将\ascii{Orientation}设计为值对象。此处用了一个小技巧,通过\ascii{Orientation}的构造函数,将它自动地注册给了\ascii{orientations}数组中。

\begin{leftbar}
\begin{c++}[caption={src/robot-cleaner/Orientation.cpp}]
namespace
{
    Orientation* orientations[4] = { nullptr };

    inline int numOfTurnTo(bool left)
    { return left ? 3 : 1; }
}

Orientation& Orientation::turnTo(bool left) const
{
    return *orientations[(order + numOfTurnTo(left)) % 4];
}

inline Orientation::Orientation(int order)
  : order(order)
{
    orientations[order] = this;
}

#define DEF_ORIENTATION(orientation, order) \
const Orientation Orientation::orientation(order);

DEF_ORIENTATION(east,  0)
DEF_ORIENTATION(south, 1)
DEF_ORIENTATION(west,  2)
DEF_ORIENTATION(north, 3)
\end{c++}
\end{leftbar}

保证测试通过后,再将\ascii{Position}中的\ascii{TurnTo}的逻辑委托给\ascii{Orientation}类,然后将\ascii{Position.cpp}冗余的代码删除掉,确保测试通过。

\begin{tabular}{@{}p{\dimexpr0.45\linewidth-\tabcolsep} 
                 | p{\dimexpr0.45\linewidth-\tabcolsep}@{}}
\begin{c++}[caption={src/robot-cleaner/Position.cpp}]
namespace
{
    Orientation orientations[] = { EAST, SOUTH, WEST, NORTH };
    
    inline int numOfTurnTo(bool left)
    { return left ? 3 : 1; }
}

void Position::turnTo(bool left)
{
    orientation = orientations[(orientation + numOfTurnTo(left)) % 4];
}
\end{c++}
&
\begin{c++}[caption={src/robot-cleaner/Position.cpp}]
void Position::turnTo(bool left)
{
    orientation = orientation.turnTo(left);
}
\end{c++}
\end{tabular}

测试通过,重构完毕。

\subsubsection{搬迁\ascii{MoveOn}}

先为\ascii{Orientation}对象增加\ascii{offset}字段,当执行\ascii{forward}时,用于表示在\ascii{X}轴移动的单位数目。

\begin{tabular}{@{}p{\dimexpr0.45\linewidth-\tabcolsep} 
                 | p{\dimexpr0.45\linewidth-\tabcolsep}@{}}
\begin{c++}[caption={src/robot-cleaner/Orientation.cpp}]
#include "robot-cleaner/Orientation.h"

namespace
{
    Orientation* orientations[4] = { nullptr };
}

inline Orientation::Orientation(int order)
  : order(order)
{
    orientations[order] = this;
}

#define DEF_ORIENTATION(orientation, order) \
const Orientation Orientation::orientation(order);

DEF_ORIENTATION(east,  0)
DEF_ORIENTATION(south, 1)
DEF_ORIENTATION(west,  2)
DEF_ORIENTATION(north, 3)
\end{c++}
&
\begin{c++}[caption={src/robot-cleaner/Orientation.cpp}]
#include "robot-cleaner/Orientation.h"

namespace
{
    Orientation* orientations[4] = { nullptr };
}

inline Orientation::Orientation(int order, int offset)
  : order(order), offset(offset)
{
    orientations[order] = this;
}

#define DEF_ORIENTATION(orientation, order, offset) \
const Orientation Orientation::orientation(order, offset);

DEF_ORIENTATION(east,  0, 1)
DEF_ORIENTATION(south, 1, 0)
DEF_ORIENTATION(west,  2, -1)
DEF_ORIENTATION(north, 3, 0)
\end{c++}
\end{tabular}

将\ascii{MoveOn}的逻辑从\ascii{Position}搬迁到\ascii{Orientation}中来,快速通过测试用例。

\begin{leftbar}
\begin{c++}[caption={src/robot-cleaner/Orientation.cpp}]
inline int Orientation::diff(int order) const
{ 
    return orientations[order]->offset; 
}

inline int Orientation::xOffset() const
{ 
    return diff(order); 
}

inline int Orientation::yOffset() const
{ 
    return diff(3-order); 
}

void Orientation::moveOn(int& x, int& y, int step) const
{
    x += step*xOffset();
    y += step*yOffset();
}
\end{c++}
\end{leftbar}

其中,\ascii{Orientation::moveOn}表示将\ascii{(x, y)}沿着该方向移动\ascii{step}。

测试通过后,将\ascii{Position::moveOn}委托给\ascii{Orientation::moveOn},确保测试通过后,删除\ascii{Position}中无用的代码。

\begin{tabular}{@{}p{\dimexpr0.45\linewidth-\tabcolsep} 
                 | p{\dimexpr0.45\linewidth-\tabcolsep}@{}}
\begin{c++}[caption={src/robot-cleaner/Position.cpp}]
namespace
{
    const int offsets[] = { 1, 0, -1, 0 };
}

inline int Position::xOffset()
{ 
    return offsets[orientation]; 
}

inline int Position::yOffset()
{ 
    return offsets[3 - orientation]; 
}

void Position::moveOn(int step)
{
    x += step*xOffset();
    y += step*yOffset();
}
\end{c++}
&
\begin{c++}[caption={src/robot-cleaner/Position.cpp}]
void Position::moveOn(int step)
{
    orientation.moveOn(x, y, step);
}
\end{c++}
\end{tabular}

测试通过,重构完毕。最后将\ascii{Orientation::operator int}删除,将其完全升级为安全的枚举类型,确保测试通过。

\subsubsection{抽取Point}

设计\ascii{Orientation::moveOn}时,传递了基本数据类型\ascii{(x, y)},这显然是不明智的,需要进一步处理。通过分析,其实\ascii{(x, y)}代表的是\ascii{Robot}的坐标信息,在此提取一个\ascii{Point}概念。

先建立了\ascii{Point}的概念,暂时将成员变量公开,以便测试尽快通过。

\begin{leftbar}
\begin{c++}[caption={include/robot-cleaner/Point.h}]
#ifndef H65A54D80_942C_4AB0_846B_A0568EA5200D
#define H65A54D80_942C_4AB0_846B_A0568EA5200D
    
#include "base/EqHelper.h"  
  
sturct Point
{
    Point(int x, int y);
    
    DECL_EQUALS(Point);
    
    int x;
    int y;
};

#endif
\end{c++}
\end{leftbar}

\begin{leftbar}
\begin{c++}[caption={src/robot-cleaner/Point.cpp}]
#include "robot-cleaner/Point.cpp"

Point::Point(int x, int y)
  : x(x), y(y)
{}

DEF_EQUALS(Point)
{
    return FIELD_EQ(x) && FIELD_EQ(y);
}

\end{c++}
\end{leftbar}

测试通过后,将\ascii{Point}的概念加入到\ascii{Position}类中,让所有测试运行通过。

\begin{tabular}{@{}p{\dimexpr0.45\linewidth-\tabcolsep} 
                 | p{\dimexpr0.45\linewidth-\tabcolsep}@{}}
\begin{c++}[caption={src/robot-cleaner/Position.cpp}]  
Position::Position(int x, int y, const Orientation& orientation)
  : x(x), y(y), orientation(orientation)
{}
  
void Position::moveOn(int step)
{
    orientation.moveOn(x, y, step);
}

DEF_EQUALS(Position)
{
    return FIELD_EQ(x) && FIELD_EQ(y) && FIELD_EQ(orientation);
}
\end{c++}
&
\begin{c++}[caption={src/robot-cleaner/Position.cpp}]  
Position::Position(int x, int y, const Orientation& orientation)
  : point(x, y), orientation(orientation)
{}
  
void Position::moveOn(int step)
{
    orientation.moveOn(point, step);
}

DEF_EQUALS(Position)
{
    return FIELD_EQ(point) && FIELD_EQ(orientation);
}
\end{c++}
\end{tabular}

其中,\ascii{orientation.moveOn(point, step)}表示:将\ascii{point}沿着\ascii{orientation}方向移动\ascii{step}步,但表达力明显下降。如果换成\ascii{point.moveOn(step, orientation)},则更能体现这个含义。暂时这样处理,接下来再做进一步处理。

接下来,为了修正编译错误,尽快恢复测试,再将\ascii{Orientation::moveOn}的入参换为\ascii{Point}类型。

\begin{tabular}{@{}p{\dimexpr0.45\linewidth-\tabcolsep} 
                 | p{\dimexpr0.45\linewidth-\tabcolsep}@{}}
\begin{c++}[caption={src/robot-cleaner/Orientation.cpp}]
void Orientation::moveOn(int& x, int& y, int step) const
{
    x += step*xOffset();
    y += step*yOffset();
}
\end{c++}
&
\begin{c++}[caption={src/robot-cleaner/Orientation.cpp}]
void Orientation::moveOn(Point& point, int step) const
{
    point.x += step*xOffset();
    point.y += step*yOffset();
}
\end{c++}
\end{tabular}

测试通过后,再将\ascii{Orientation::moveOn}中运算搬迁到\ascii{Point::move}中去。

\begin{tabular}{@{}p{\dimexpr0.45\linewidth-\tabcolsep} 
                 | p{\dimexpr0.45\linewidth-\tabcolsep}@{}}
\begin{c++}[caption={src/robot-cleaner/Orientation.cpp}]
void Orientation::moveOn(Point& point, int step) const
{
    point.x += step*xOffset();
    point.y += step*yOffset();
}
\end{c++}
&
\begin{c++}[caption={src/robot-cleaner/Orientation.cpp}]
void Orientation::moveOn(Point& point, int step) const
{
    point.move(step, xOffset(), yOffset());
}
\end{c++}
\end{tabular}

而\ascii{Point}增加\ascii{move}接口,完成最后的偏移运算。

\begin{leftbar}
\begin{c++}[caption={src/robot-cleaner/Point.cpp}]
void Point::move(int step, int xOffset, int yOffset)
{
    x += step*xOffset;
    y += step*yOffset;
}
\end{c++}
\end{leftbar}

测试通过后,再将\ascii{Point::x, Point::y}隐藏起来,保证测试通过。

\begin{tabular}{@{}p{\dimexpr0.45\linewidth-\tabcolsep} 
                 | p{\dimexpr0.45\linewidth-\tabcolsep}@{}}
\begin{c++}[caption={include/robot-cleaner/Point.h}]  
#ifndef H65A54D80_942C_4AB0_846B_A0568EA5200D
#define H65A54D80_942C_4AB0_846B_A0568EA5200D

#include "base/EqHelper.h"

struct Point
{
    Point(int x, int y);

    void move(int xOffset, int yOffset);

    __DECL_EQUALS(Point);

    int x;
    int y;
};

#endif
\end{c++}
&
\begin{c++}[caption={include/robot-cleaner/Point.h}]  
#ifndef H65A54D80_942C_4AB0_846B_A0568EA5200D
#define H65A54D80_942C_4AB0_846B_A0568EA5200D

#include "base/EqHelper.h"

struct Point
{
    Point(int x, int y);

    void move(int xOffset, int yOffset);

    __DECL_EQUALS(Point);

private:
    int x, y;
};

#endif
\end{c++}
\end{tabular}

测试通过了,此刻彻底消除了“方向顺序”的知识在三个地方存在三种表示的重复设计,将三张表规整为一张表。不仅在物理位置上的统一,也将归属于自身的职责也搬迁过来了,知识更加内聚。

\begin{leftbar}
\begin{c++}[caption={src/robot-cleaner/Orientation.cpp}]
DEF_ORIENTATION(east,  0, 1)
DEF_ORIENTATION(south, 1, 0)
DEF_ORIENTATION(west,  2, -1)
DEF_ORIENTATION(north, 3, 0)
\end{c++}
\end{leftbar}

\subsection{依恋情节}

至此,还存在几个明显的坏味道有待进一步改善。

\begin{enum}
\eitem{从职责归属关系看,让\ascii{Position}承担\ascii{turnTo, moveOn}的语义有些晦涩;}
\eitem{\ascii{Position}与\ascii{Orientation}之间存在不必要的依赖关系,两者因为增加新的指令都不能独立变化,未能做到彻底的修改的封闭性。}
\eitem{\ascii{Instruction}的\ascii{turnTo},\ascii{moveOn}实现具有副作用。}
\end{enum}

\subsubsection{TDA}

首先,修改\ascii{Instruction}的接口。

\begin{tabular}{@{}p{\dimexpr0.45\linewidth-\tabcolsep} 
                 | p{\dimexpr0.45\linewidth-\tabcolsep}@{}}
\begin{c++}[caption={include/robot-cleaner/Instruction.h}]  
struct Position;

DEFINE_ROLE(Instruction)
{
    ABSTRACT(void exec(Position&));
};
\end{c++}
&
\begin{c++}[caption={include/robot-cleaner/Instruction.h}] 
struct Point;
struct Orientation;  
 
DEFINE_ROLE(Instruction)
{
    ABSTRACT(void exec(Point&, Orientation&));
};
\end{c++}
\end{tabular}

先修改\ascii{RobotCleaner::exec}的实现。

\begin{tabular}{@{}p{\dimexpr0.45\linewidth-\tabcolsep} 
                 | p{\dimexpr0.45\linewidth-\tabcolsep}@{}}
\begin{c++}[caption={src/robot-cleaner/RobotCleaner.cpp}]  
void RobotCleaner::exec(Instruction* instruction)
{
    instruction->exec(position);
    delete instruction;
}
\end{c++}
&
\begin{c++}[caption={src/robot-cleaner/RobotCleaner.cpp}]  
void RobotCleaner::exec(Instruction* instruction)
{
    instruction->exec(point.getPoint(), orientation.getOrientation());
    delete instruction;
}
\end{c++}
\end{tabular}

最后删除\ascii{Position}的\ascii{turnTo, moveOn}接口,并为其增加新的接口。

\begin{leftbar}
\begin{c++}[caption={src/robot-cleaner/Position.cpp}]  
Point& Position::getPoint()
{
    return point;
}

Orientation& Position::getOrientation()
{
    return orientation;
}
\end{c++}
\end{leftbar}

接下来,修改和实现\ascii{Instruction}所有的子类实现。

\begin{tabular}{@{}p{\dimexpr0.45\linewidth-\tabcolsep} 
                 | p{\dimexpr0.45\linewidth-\tabcolsep}@{}}
\begin{c++}[caption={src/robot-cleaner/Instruction.cpp}]  
namespace
{
    struct EmptyInstruction : Instruction
    {
        OVERRIDE(void exec(Position& position)) {}
    };

    struct TurnToInstruction : Instruction
    {
        explicit TurnToInstruction(bool left)
          : left(left)
        {}

    private:
        OVERRIDE(void exec(Position& position))
        { 
            position.turnTo(left);
        }

    private:
        bool left;
    };

    struct MoveOnInstruction : Instruction
    {
        explicit MoveOnInstruction(bool forward, int n)
          : step(prefix(forward)*n)
        {}

    private:
        OVERRIDE(void exec(Position& position))
        { 
            position.moveOn(step); 
        }

    private:
        static int prefix(bool forward)
        { 
            return forward ? 1 : -1; 
        }

    private:
        int step;
    };

    struct RepeatedInstruction : Instruction
    {
        RepeatedInstruction(Instruction* instruction, int n)
         : instruction(instruction), n(n)
        {}

    private:
        OVERRIDE(void exec(Position& position))
        {
            for (auto i=0; i<n; i++)
            {
                instruction->exec(position);
            }
        }

    private:
        std::unique_ptr<Instruction> instruction;
        int n;
    };

    struct SequentialInstruction : Instruction
    {
        SequentialInstruction(std::initializer_list<Instruction*> instructions)
          : instructions(instructions)
        {}

        ~SequentialInstruction()
        {
            for(auto instruction : instructions)
            {
                delete instruction;
            }
        }

    private:
        OVERRIDE(void exec(Position& position))
        {
            for(auto instruction : instructions)
            {
                instruction->exec(position);
            }
        }

    private:
        std::vector<Instruction*> instructions;
    };
}
\end{c++}
&
\begin{c++}[caption={src/robot-cleaner/Instruction.cpp}]  
namespace
{
    struct EmptyInstruction : Instruction
    {
        OVERRIDE(void exec(Point&, Orientation&)) {}
    };

    struct TurnToInstruction : Instruction
    {
        explicit TurnToInstruction(bool left)
          : left(left)
        {}

    private:
        OVERRIDE(void exec(Point&, Orientation& orientation))
        {
            orientation = orientation.turnTo(left);
        }

    private:
        bool left;
    };

    struct MoveOnInstruction : Instruction
    {
        explicit MoveOnInstruction(bool forward, int n)
          : step(prefix(forward)*n)
        {}

    private:
        OVERRIDE(void exec(Point& point, Orientation& orientation))
        { 
            orientation.moveOn(point, step); 
        }

    private:
        static int prefix(bool forward)
        { 
            return forward ? 1 : -1;
        }

    private:
        int step;
    };

    struct RepeatedInstruction : Instruction
    {
        RepeatedInstruction(Instruction* instruction, int n)
         : instruction(instruction), n(n)
        {}

    private:
        OVERRIDE(void exec(Point& point, Orientation& orientation))
        {
            for (auto i=0; i<n; i++)
            {
                instruction->exec(point, orientation);
            }
        }

    private:
        std::unique_ptr<Instruction> instruction;
        int n;
    };

    struct SequentialInstruction : Instruction
    {
        SequentialInstruction(std::initializer_list<Instruction*> instructions)
          : instructions(instructions)
        {}

        ~SequentialInstruction()
        {
            for(auto instruction : instructions)
            {
                delete instruction;
            }
        }

    private:
        OVERRIDE(void exec(Point& point, Orientation& orientation))
        {
            for(auto instruction : instructions)
            {
                instruction->exec(point, orientation);
            }
        }

    private:
        std::vector<Instruction*> instructions;
    };
}
\end{c++}
\end{tabular}

最后删除\ascii{Position}的\ascii{turnTo, moveOn}接口,保证测试通过。
  
测试通过了,此刻\ascii{Position}与\ascii{Orientation}之间,及其\ascii{Position}与新增\ascii{Instruction}之间的耦合度被降低了;另外,\ascii{Position}也不承担晦涩的\ascii{turnTo, moveOn}的职责。

这是\ascii{TDA, Tell, Don't Ask}原则一个具体运用,但也为此付出了另外的代价:\ascii{Position}不得不提供两个\ascii{get}接口,而且是可修改的引用类型。

\subsubsection{改善表达力}

\eitem{\ascii{MoveInstruction}实现\ascii{moveOn}时,委托给了\ascii{Orientation},让\ascii{Orientation}承担\ascii{moveOn}的职责有些晦涩,应该将两者的语义换一下。}

\begin{tabular}{@{}p{\dimexpr0.45\linewidth-\tabcolsep} 
                 | p{\dimexpr0.45\linewidth-\tabcolsep}@{}}
\begin{c++}[caption={src/robot-cleaner/Instruction.cpp}]  
struct MoveOnInstruction : Instruction
{
    explicit MoveOnInstruction(bool forward, int n)
      : step(prefix(forward)*n)
    {}

private:
    OVERRIDE(void exec(Point& point, Orientation& orientation))
    { 
        orientation.moveOn(point, step); 
    }

private:
    static int prefix(bool forward)
    { 
        return forward ? 1 : -1;
    }

private:
    int step;
};
\end{c++}
&
\begin{c++}[caption={src/robot-cleaner/Instruction.cpp}]  
struct MoveOnInstruction : Instruction
{
    explicit MoveOnInstruction(bool forward, int n)
      : step(prefix(forward)*n)
    {}

private:
    OVERRIDE(void exec(Point& point, Orientation& orientation))
    { 
        point = point.moveOn(step, orientation);
    }

private:
    static int prefix(bool forward)
    { 
        return forward ? 1 : -1;
    }

private:
    int step;
};
\end{c++}
\end{tabular}

\ascii{point.moveOn}表示\ascii{point}沿着\ascii{orientation}方向移动了\ascii{step}步后到了一个新的位置。

然后将\ascii{Point::move}重构为\ascii{Point::moveOn},保证测试通过。

\begin{tabular}{@{}p{\dimexpr0.45\linewidth-\tabcolsep} 
                 | p{\dimexpr0.45\linewidth-\tabcolsep}@{}}
\begin{c++}[caption={src/robot-cleaner/Point.cpp}]
void Point::move(int step, int xOffset, int yOffset)
{
    x += step*xOffset;
    y += step*yOffset;
}
\end{c++}
&
\begin{c++}[caption={src/robot-cleaner/Point.cpp}]
Point Point::moveOn(int step, const Orientation& orientation)
{
    return Point( x+step*orientation.getXOffset()
                , y+step*orientation.getYOffset());
}
\end{c++}
\end{tabular}

但需要暴露\ascii{Orientation::getXOffset,Orientation::getYOffset}。这不是一个很好的设计,可以将\ascii{Point::x, Point::y}传递给\ascii{Orientation},根据\ascii{Orientation}自身所处的方向,将\ascii{x, y}向前移动\ascii{step},并返回一个新的\ascii{Point}位置。

\begin{tabular}{@{}p{\dimexpr0.45\linewidth-\tabcolsep} 
                 | p{\dimexpr0.45\linewidth-\tabcolsep}@{}}
\begin{c++}[caption={src/robot-cleaner/Point.cpp}]
Point Point::moveOn(int step, const Orientation& orientation)
{
    return Point( x+step*orientation.getXOffset()
                , y+step*orientation.getYOffset());
}
\end{c++}
&
\begin{c++}[caption={src/robot-cleaner/Point.cpp}]
Point Point::moveOn(int step, const Orientation& orientation)
{
    return orientation.moveOn(x, y, step);
}
\end{c++}
\end{tabular}

最后重构实现\ascii{Orientation::moveOn}的逻辑。

\begin{tabular}{@{}p{\dimexpr0.45\linewidth-\tabcolsep} 
                 | p{\dimexpr0.45\linewidth-\tabcolsep}@{}}
\begin{c++}[caption={src/robot-cleaner/Orientation.cpp}]
void Orientation::moveOn(Point& point, int step) const
{
    point.move(step, xOffset(), yOffset());
}
\end{c++}
&
\begin{c++}[caption={src/robot-cleaner/Orientation.cpp}]
Point Orientation::moveOn(int x, int y, int step) const
{
    return Point(x+step*xOffset(), y+step*yOffset());
}
\end{c++}
\end{tabular}

测试通过了。接下来将\ascii{Position}提供的两个\ascii{get}接口消除掉,使用多重继承更好可以表达它们之间的聚合关系。

\begin{tabular}{@{}p{\dimexpr0.45\linewidth-\tabcolsep} 
                 | p{\dimexpr0.45\linewidth-\tabcolsep}@{}}
\begin{c++}[caption={src/robot-cleaner/Orientation.cpp}]
#include "robot-cleaner/Point.h"
#include "robot-cleaner/Orientation.h"

struct Position
{
    Position(int x, int y, const Orientation& orientation);

    DECL_EQUALS(Position);
    
    Point& getPoint();
    Orientation& getOrientation();
    
private:
    Point point;
    Orientation orientation;
};
\end{c++}
&
\begin{c++}[caption={src/robot-cleaner/Orientation.cpp}]
#include "robot-cleaner/Point.h"
#include "robot-cleaner/Orientation.h"

struct Position : Point, Orientation
{
    Position(int x, int y, const Orientation& orientation);

    DECL_EQUALS(Position);
};
\end{c++}
\end{tabular}

先修改\ascii{RobotCleaner::exec}的实现。

\begin{tabular}{@{}p{\dimexpr0.45\linewidth-\tabcolsep} 
                 | p{\dimexpr0.45\linewidth-\tabcolsep}@{}}
\begin{c++}[caption={src/robot-cleaner/RobotCleaner.cpp}]  
void RobotCleaner::exec(Instruction* instruction)
{
    instruction->exec(point.getPoint(), orientation.getOrientation());
    delete instruction;
}
\end{c++}
&
\begin{c++}[caption={src/robot-cleaner/RobotCleaner.cpp}]  
void RobotCleaner::exec(Instruction* instruction)
{
    instruction->exec(SELF(position, Point), SELF(position, Position));
    delete instruction;
}
\end{c++}
\end{tabular}

测试通过了,重构完毕。

\end{content}

\section{最后的话}

\begin{content}

首先看\ascii{Instruction}的设计,它将指令执行的算法封装起来,独立地随这指令集变化而变化;并且提供了工厂方法,对外提供\ascii{DSL},将它的子类实现全部隐藏到了实现文件中,对外暴露的只有\ascii{Instruction}的抽象接口,在提高表达力的同时,增强了信息隐藏的机制。

\begin{leftbar}
\begin{c++}[caption={test/robot-cleaner/TestRobotCleaner.h}]
#ifndef HC37C3D94_43F1_4677_BD56_34CB78EFEC75
#define HC37C3D94_43F1_4677_BD56_34CB78EFEC75

#include "infra/base/Role.h"
#include <initializer_list>

struct Point;
struct Orientation;

DEFINE_ROLE(Instruction)
{
    ABSTRACT(void exec(Point& point, Orientation& orientation) const);
};

Instruction* left();
Instruction* right();

Instruction* forward_n(int n);
Instruction* backward_n(int n);
Instruction* repeat(Instruction*, int n);

inline Instruction* forward()
{ return forward_n(1);  }

inline Instruction* backward()
{ return backward_n(1); }

inline Instruction* round()
{ return repeat(right(), 2); }

Instruction* __sequential(std::initializer_list<Instruction*>);

#define sequential(...) __sequential({ __VA_ARGS__ })

#endif
\end{c++}
\end{leftbar}

再来看\ascii{Position}与\ascii{Point},\ascii{Orientation}之间的关系。\ascii{Position}是由\ascii{Point}和\ascii{Orientation}聚合而成的。\ascii{Point}具有\ascii{MoveOn}的语义,而\ascii{Orientation}具有\ascii{TurnTo}的语义;另外,\ascii{Point::moveOn}时,是以\ascii{Orientation}的方向来决定该怎么移动的,所以\ascii{Orientation}将协作\ascii{Point}完成\ascii{MoveOn}的移动过程。

接着看\ascii{Instruction}与\ascii{Point},\ascii{Orientation}的关系。\ascii{TurnToInstruction}将于\ascii{Orientation}交互完成转向;\ascii{MoveOnInstruction}将与\ascii{Point}交互,在\ascii{Orientation}的协作下完成移动。

需要强调的是,\ascii{Instruction}与\ascii{Position}没有任何关系,它在领域内只是对\ascii{Point},\ascii{Orientation}聚合而已。

最后重要的是,\ascii{TurnLeft}与\ascii{TurnRight}之间,\ascii{Forward}与\ascii{Backward}之间,犹如正反两方面。如果存在\ascii{TurnLeftInstrction}, \ascii{TurnRightInstructin}两个概念,并且它们的逻辑实现没有依赖一个更本质的抽象,则它们之间是存在重复设计的。

我们将\ascii{TurnLeftInstrction}, \ascii{TurnRightInstructin}两个概念合并为\ascii{TurnToInstruction},而将\ascii{ForwardInstrction}, \ascii{BackwardInstructin}两个概念合并为\ascii{MoveOnInstruction},采用了简单\ascii{bool}接口消除它们逻辑上的重复。

当然,此处\ascii{bool}接口的使用没有影响用户友好性,因为用户使用的是抽象了的、友好性的\ascii{DSL}接口,例如\ascii{left()}或者\ascii{right()}指令;另外,调用字面值\ascii{true/false}时,与定义实现并不远,也不存在晦涩难懂,需要用户跳转代码等难题。

\end{content}
