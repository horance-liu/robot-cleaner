\begin{savequote}[45mm]  
\ascii{Premature optimization is the root of all evil.} 
\qauthor{\ascii{– Donald Knuth}}
\end{savequote}
  
\chapter{MoveOn(n)} 
\label{ch:move-on-n}

\section{需求}

\begin{content}

\begin{enum}
\eitem{当\ascii{Robot}收到\ascii{FORWARD(n)}指令后,\ascii{Robot}保持方向,向前移动\ascii{n}个坐标;}
\eitem{当\ascii{Robot}收到\ascii{BACKWARD(n)}指令后,\ascii{Robot}保持方向,向后移动\ascii{n}个坐标;}
\eitem{其中\ascii{n}在\ascii{[1..10]之间};}
\end{enum}

例如:
\begin{enum}
\eitem{\ascii{Robot}原来在\ascii{(0, 0, N)},执行\ascii{FORWARD(10)}指令后,新的位置在\ascii{(0, 10, N)};}
\eitem{\ascii{Robot}原来在\ascii{(0, 0, N)},执行\ascii{BACKWARD(10)}指令后,新的位置在\ascii{(0, -10, N)};}
\end{enum}

\end{content}

\section{forward\_n指令}

\begin{content}

\subsection{测试用例}

\begin{leftbar}
\begin{c++}[caption={test/robot-cleaner/TestRobotCleaner.h}]
TEST("forward(n) instruction")
{
    WHEN_I_send_instruction(forward_n(10));
    THEN_the_robot_cleaner_should_be_in(Position(0, 10, NORTH));
}
\end{c++}
\end{leftbar}

\subsection{实现forward\_n指令}

我很有自信地、大胆地将\ascii{forward}委托给\ascii{forward\_n},步子虽然有点大,而且冒着让已有的用例失败的可能性,但就问题域而言,还在自己控制范围之内。

\begin{leftbar}
\begin{c++}[caption={include/robot-cleaner/Instruction.h}]
#ifndef HC37C3D94_43F1_4677_BD56_34CB78EFEC75
#define HC37C3D94_43F1_4677_BD56_34CB78EFEC75

#include "base/Role.h"

struct Position;

DEFINE_ROLE(Instruction)
{
    ABSTRACT(void exec(Position& position));
};

Instruction* forward_n(int n);

inline Instruction* forward()
{ return forward_n(1); }

Instruction* backward();

#endif
\end{c++}
\end{leftbar}

为了保证即有测试通过,此处\ascii{backward}的实现,进行了简单的处理。

\begin{leftbar}
\begin{c++}[caption={src/robot-cleaner/Instruction.cpp}]
namespace
{
    struct MoveOnInstruction : Instruction
    {
        explicit MoveOnInstruction(int n, bool forward)
          : step(prefix(forward)*n)
        {}

    private:
        OVERRIDE(void exec(Position& position))
        {
            position.moveOn(step);
        }

    private:
        static int prefix(bool forward)
        {
            return forward ? 1 : -1;
        }

    private:
        int step;
    };
}

Instruction* forward_n(int n)
{ return new MoveOnInstruction(n, true); }

Instruction* backward()
{ return new MoveOnInstruction(1, false); }
\end{c++}
\end{leftbar}

\end{content}

\section{backward\_n指令}

\begin{content}
    
\subsection{测试用例}

\begin{leftbar}
\begin{c++}[caption={test/robot-cleaner/TestRobotCleaner.h}]
TEST("forward(n) instruction")
{
    WHEN_I_send_instruction(backward_n(10));
    THEN_the_robot_cleaner_should_be_in(Position(0, -10, NORTH));
}
\end{c++}
\end{leftbar}    

\subsection{实现backward\_n指令}

同样的道理,将\ascii{backward}内联,委托给\ascii{backward\_n}指令。

\begin{leftbar}
\begin{c++}[caption={include/robot-cleaner/Instruction.h}]
#ifndef HC37C3D94_43F1_4677_BD56_34CB78EFEC75
#define HC37C3D94_43F1_4677_BD56_34CB78EFEC75

#include "base/Role.h"

struct Position;

DEFINE_ROLE(Instruction)
{
    ABSTRACT(void exec(Position& position));
};

Instruction* forward_n(int n);
Instruction* backward_n(int n);

inline Instruction* forward()
{ return forward_n(1); }

inline Instruction* backward()
{ return backward_n(1); }

#endif
\end{c++}
\end{leftbar}

在将\ascii{backward\_n}显式实现。

\begin{leftbar}
\begin{c++}[caption={src/robot-cleaner/Instruction.cpp}]
namespace
{
    struct MoveOnInstruction : Instruction
    {
        explicit MoveOnInstruction(int n, bool forward)
          : step(prefix(forward)*n)
        {}

    private:
        OVERRIDE(void exec(Position& position))
        {
            position.moveOn(step);
        }

    private:
        static int prefix(bool forward)
        {
            return forward ? 1 : -1;
        }

    private:
        int step;
    };
}  
  
Instruction* forward_n(int n)
{
    return new MoveOnInstruction(n, true);
}

Instruction* backward_n(int n)
{
    return new MoveOnInstruction(n, false);
}
\end{c++}
\end{leftbar}

\end{content}

\section{边界}

\begin{content}

\subsection{测试用例}

\begin{leftbar}
\begin{c++}[caption={src/robot-cleaner/Instruction.cpp}]
TEST("forward(n) instruction: n out of bound")
{
    WHEN_I_send_instruction(forward_n(11));
    THEN_the_robot_cleaner_should_be_in(Position(0, 0, NORTH));
}

TEST("backward(n) instruction: n out of bound")
{
    WHEN_I_send_instruction(backward_n(11));
    THEN_the_robot_cleaner_should_be_in(Position(0, 0, NORTH));
}
\end{c++}
\end{leftbar}

\subsection{快速实现}

\begin{tabular}{@{}p{\dimexpr0.45\linewidth-\tabcolsep} 
                 | p{\dimexpr0.45\linewidth-\tabcolsep}@{}}
\begin{c++}[caption={src/robot-cleaner/Instruction.cpp}]
namespace
{
    struct MoveOnInstruction : Instruction
    {
        explicit MoveOnInstruction(int n, bool forward)
          : step(prefix(forward)*n)
        {}

    private:
        OVERRIDE(void exec(Position& position))
        {
            position.moveOn(step);
        }

    private:
        static int prefix(bool forward)
        {
            return forward ? 1 : -1;
        }
        
    private:
        int step;
    };
}  
  
Instruction* forward_n(int n)
{
    return new MoveOnInstruction(n, true);
}

Instruction* backward_n(int n)
{
    return new MoveOnInstruction(n, false);
}
\end{c++}
&
\begin{c++}[caption={src/robot-cleaner/Instruction.cpp}]
namespace
{
    const int MIN_MOVE_NUM = 1;
    const int MAX_MOVE_NUM = 10;
    
    struct MoveOnInstruction : Instruction
    {
        explicit MoveOnInstruction(int n, bool forward)
          : step(prefix(forward)*protect(n))
        {}

    private:
        OVERRIDE(void exec(Position& position))
        {
            position.moveOn(step);
        }

    private:
        static int prefix(bool forward)
        {
            return forward ? 1 : -1;
        }
        
        static int protect(int n)
        {
            return (MIN_MOVE_NUM <= n  
              && n <= MAX_MOVE_NUM) ? n : 0;
        }

    private:
        int step;
    };
}  
  
Instruction* forward_n(int n)
{
    return new MoveOnInstruction(n, true);
}

Instruction* backward_n(int n)
{
    return new MoveOnInstruction(n, false);
}
\end{c++}
\end{tabular}

测试通过了,但此处处理的手段还不够光彩,有待进一步改善。
    
\end{content}
