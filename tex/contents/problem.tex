\begin{savequote}[45mm]
\ascii{Do the simplest thing that could possibly work.}
\qauthor{\ascii{- Kent Beck}}
\end{savequote}

\chapter{Background} 
\label{ch:background}

\section{问题提出}

\begin{content}

扫地机器人\ascii{(robot cleaner)},又称自动打扫机、智能吸尘器等。能凭借一定的人工智能,自动在房间内完成地板清理工作。

\ascii{Bosch}公司是领先的扫地机器人制造商。\ascii{Bosch}的工程师研发了一款机器人,它接受远端遥感指令,并完成一些简单的动作。

为了方便控制机器人的导航,工程师使用三元组\ascii{(x, y, d)}来表示机器人的位置信息;其中\ascii{(x,y)}表示机器人的坐标位置,\ascii{d}表示机器人的方向(包括\ascii{East, South, West, North}四个方向)。

假设机器人初始位置为\ascii{(0, 0, N)},表示机器人处在原点坐标、并朝向北。

\end{content}

\section{技能要求}

\begin{content}

在实现需求的前提下,增加如下一些技能要求。

\begin{enum}
\eitem{每个迭代\ascii{45}分钟}
\eitem{结对编程}
\eitem{不允许使用鼠标}
\eitem{不允许使用\ascii{IDE}}
\eitem{坚持使用\ascii{TDD}开发}
\eitem{持续重构代码}
\eitem{坚持良好的代码提交习惯}
\eitem{教练点评的坏味道,下一迭代必须纠正}
\eitem{参与互相点评的环节}
\eitem{互相重构不同\ascii{pair}的代码}
\end{enum}

\end{content}

\section{参考实现}

\begin{content}

本文所使用的代码,读者可以从\ascii{Gitlab}上自由下载、修改和传播。

软件设计是一门艺术,因作者的经验、能力都非常有限,此实现仅作供参考。如果大家发现了问题,或者有更加简单、漂亮的设计,请及时反馈给我,我将感激不尽。

\ascii{Git}地址:\href{https://gitlab.com/horance/robot-cleaner-cpp}{git clone https://gitlab.com/horance/robot-cleaner-cpp.git}

\ascii{Email}地址:\href{mailto:horance@outlook.com}{horance@outlook.com}

\end{content}
